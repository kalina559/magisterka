\newpage % Rozdziały zaczynamy od nowej strony.
\section{Wstęp}
\subsection{Wprowadzenie}
Gry komputerowe od dekad stanowią istotny element rozrywki, rozwijając się od prostych gier zręcznościowych po skomplikowane symulacje otwartych światów. W miarę postępu technologicznego, role komputerów w tworzeniu gier komputerowych stawały się coraz bardziej złożone. Początkowo komputery były odpowiedzialne jedynie za przetwarzanie prostych algorytmów, które definiowały zasady gry i interakcje między elementami na ekranie. Z czasem jednak, wraz z rozwojem komputerów, pojawiła się możliwość implementacji bardziej zaawansowanych mechanizmów decyzyjnych, które pozwoliły na podejmowanie złożonych decyzji w czasie rzeczywistym.

Jednym z kluczowych elementów, które umożliwiły rozwój gier, jest algorytmika. Algorytmy, jako zbiory jasno zdefiniowanych kroków do rozwiązania problemu, są podstawą każdego programu komputerowego, w tym także gier. W grach komputerowych algorytmy decyzyjne są szczególnie istotne, ponieważ determinują, jak komputerowy przeciwnik będzie reagował na działania gracza. Złożoność tych algorytmów może wahać się od prostych drzewek decyzyjnych do skomplikowanych systemów sztucznej inteligencji, które naśladują ludzkie myślenie i adaptują się do strategii gracza. W wielu takich algorytmach wykorzystywane są heurystyki, w celu ograniczenia zapotrzebowania na moc obliczeniową komputera.

\indent Heurystyki, czyli uproszczone metody podejmowania decyzji, pomagają nam w codziennym życiu przy dokonywaniu różnych wyborów, nawet jeśli nie jesteśmy tego do końca świadomi. Często korzystamy z heurystyk, aby szybciej podjąć decyzję - nie mamy wtedy jednak pewności, że będzie to decyzja optymalna. "Heurystyka dostępności" jest często stosowana do szacowania prawdopodobieństwa zdarzeń, na podstawie tego jak łatwo jest nam przywołać przykłady wystąpienia tego zdarzenia - na przykład jeśli wielokrotnie słyszymy że na zaparkowany samochód spadło drzewo, będziemy unikać parkowania pod drzewem \cite{tversky74}.

\indent Heurystyki w kontekście informatyki również ułatwiają rozwiązywanie problemów, których złożoność obliczeniowa jest na tyle duża, że ma to bardzo negatywny wpływ na czas wykonywania obliczeń. Jako przykład może służyć problem komiwojażera - dla podanych miast oraz dystansów między nimi, musimy wyznaczyć najkrótszą drogę wiodącą z miasta A do B, która przechodzi przez wszystkie inne miasta. Dla małej liczby miast, problem można rozwiązać metodą brute-force, obliczając wszystkie możliwe ścieżki i wybierając najkrótszą. Jednak z każdym dodanym do zadania miastem, znacznie wzrasta liczba możliwych ścieżek - a wraz z nią czas potrzebny na obliczenia. W tym celu Job Bentley zaproponował uproszczony algorytm, który nie bierze pod uwagę wszystkich ścieżek. Zamiast tego, dla każdego kolejnego kroku, wybiera po prostu najbliższe miasto. Ostatecznie wybrana ścieżka nie zawsze będzie faktycznie tą możliwie najkrótszą, ale będzie do niej zbliżona \cite{Hjeij2023}.

\indent Niniejsza praca zagłębia się w zastosowanie heurystyk w informatyce, a konkretnie w grach komputerowych. Są one świetnym narzędziem do implementacji algorytmów decyzyjnych, z których korzystają komputerowi przeciwnicy w grze. Szczególnie we wpółczesnych grach z otwartym światem, podejmowanie decyzji przez przeciwników nie może za bardzo obciążać zasobów procesora, ponieważ są one również potrzebne do obliczeń fizyki gry, symulacji otaczającego gracza świata, logiki gry, etc.

\subsection{Cel pracy}
\subsubsection{Cel główny}
Analiza skuteczności algorytmów decyzyjnych przeciwnika w turowej grze strategicznej Statki.

\subsubsection{Cele pomocnicze}
\begin{itemize}
    \item Implementacja gry Statki.
    \item Udostępnienie publicznie gry, aby użytkownicy mogli ją testować.
    \item Zapewnienie możliwości starć algorytm-algorytm.
    \item Zapewnienie możliwości archiwizacji zakończonych rozgrywek w bazie danych, aby później możliwa była ich analiza.
\end{itemize}

\subsection{Metodologia}
\indent Przedmiotem badania była ocena skuteczności zaimplementowanych algorytmów. Po zakończeniu pracy powinniśmy znać odpowiedzi na pytania:
\begin{enumerate}
  \item Który algorytm decyzyjny jest najbardziej skuteczny?
  \item Co ma wpływ na skuteczność algorytmów?
  \item Czy skuteczność algorytmów różni się w zależności od tego czy gra przeciwko człowiekowi, czy innemu algorytmowi?
\end{enumerate}
Aby opowiedzieć na te pytania, przeprowadzona zostanie analiza statystyczna rozgrywek zapisanych w bazie danych.