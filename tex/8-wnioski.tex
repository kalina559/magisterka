\newpage % Rozdziały zaczynamy od nowej strony.
\section{Wnioski}

Na podstawie wyników przedstawionych w poprzednim rozdziale, możemy uszeregować algorytmy od najmniej do najbardziej skutecznego:\begin{enumerate}
    \item Algorytm losowy
    \item Rozszerzony algorytm losowy
    \item Algorytm oparty na heurystyce najbardziej prawdopodobnej lokalizacji na podstawie trafień
    \item Algorytm oparty na heurystyce maksymalizacji zysku priorytetyzującej dłuższe statki
    \item Algorytm oparty na heurystyce maksymalizacji zysku ze strzału
        \item Algorytm oparty na heurystykach maksymalizacji zysku oraz najbardziej prawdopodobnej lokalizacji na podstawie trafień priorytetyzującej dłuższe statki
    \item Algorytm oparty na heurystykach maksymalizacji zysku oraz najbardziej prawdopodobnej lokalizacji na podstawie trafień
\end{enumerate}

Na wykresie widocznym na rysunku 7.1. można zauważyć że heurystyka maksymalizacji zysku ze strzału jest bardziej skuteczna niż heurystyką najbardziej prawdopodobnej lokalizacji na podstawie trafień. Tak więc minimalizacja możliwych położeń statków na mapie jest bardziej opłacalna, nawet jeśli algorytm nie potrafi analizować trafień w celu 'dobijania' statków. Generalnie, z wykresu wyraźnie wynika, że im mniej losowości w algorytmie, tym jest skuteczniejszy - algorytm \emph{Analiza trafień} losowo wybiera ostrzeliwane komórki, jeśli nie ma żadnego trafienia, które może analizować.

Z wykresu wynika także brak znaczącej różnicy pomiędzy algorytmami priorytetyzującymi dłuższe statki, a podstawowymi wariantami tychże algorytmów. Jest to prawdopodobnie spowodowane tym, że trafienie największego statku zazwyczaj nie jest największym wyzwaniem w rozgrywce. Najbardziej problematyczne jest raczej trafienie najmniejszego statku - żadna heurystyka nie jest w stanie pomóc w odnalezieniu jego lokalizacji.

Ze zgromadzonych danych wynika też, że gdy statki mogą się ze sobą stykać, wyniki starć pomiędzy poszczególnymi algorytmami są bardziej wyrównane. Algorytmy losowe nadal są zdecydowanie najsłabsze, ale algorytmy oparte na heurystykach mają skuteczności w przedziale około 40-60\%. W przypadku tego wariantu zasad, na skuteczności wiele tracą algorytmy oparte na heurystyce maksymalizacji zysku ze strzału. Mniej ograniczeń dotyczących lokalizacji statku przez zasady, oznacza że istnieje więcej potencjalnych rozmieszczeń statków na planszy przeciwnika. Algorytm potrzebuje więcej ruchów, aby eliminować, te możliwe położenia. 

Teorię tę potwierdza analiza skuteczności wszystkich algorytmów przeciwko algorytmowi losowemu, która jest widoczna na rysunku 7.2. Łatwo zauważyć, że prawie we wszystkich przypadkach, skuteczności  znacznie różnią się zależnie od wybranych zasad. Jedynym wyjątkiem jest algorytm oparty na heurystyce najbardziej prawdopodobnej lokalizacji na podstawie trafień. Algorytm ten wybiera losowo komórki do ostrzału, jeśli na planszy nie ma żadnego trafienia - nie ma więc na niego wpływu fakt, że statki mogą się ze sobą stykać.

Różnicę dobrze widać też na rysunku 7.2, który przedstawia średnią liczbę tur w zależności od tego z jakim algorytmem mierzył się algorytm losowy. W przypadku wszystkich algorytmów poza \emph{Analizą trafień}, liczba tur gdy statki mogą się ze sobą stykać jest znacznie wyższa niż w przeciwnym wypadku. Widać też że liczba tur spada znacznie wolniej wraz ze wzrostem zaawansowania algorytmu. Wynika to z wcześniej wspomnianego faktu, że heurystyka maksymalizacji zysku ze strzału traci na efektywności gdy statki mogą się ze sobą stykać.

Z tabel 7.1-7.9 wynika, że im skuteczniejszy jest dany algorytm, tym krótsze są średnio jego rozgrywki. Średnio najmniej tur potrzebnych było w starciach algorytmów \emph{Max. zysk + analiza trafień} oraz \emph{Max. zysk rozszerzony + analiza trafień}.

W tabelach 7.11 - 7.13 widoczny jest wpływ liczby statków użytych w rozgrywce na skuteczność algorytmów.

Dodanie najmniejszych statków, składających się z tylko jednej komórki znacznie zmiejsza skuteczność bardziej złożonych algorytmów. Heurystyka \emph{Analiza trafień} jest wtedy bezużyteczna - przydaje się ona jedynie gdy chcemy wnioskować gdzie znajdują się pozostałe komórki trafionego statku. W przypadku jednomasztowców taki scenariusz nigdy nie dojdzie do skutku - trafienie równa się zatopieniu. Widać to wyraźnie w tabeli 7.11 - skuteczność przeciwko \emph{Analiza trafień} znacznie wzrosła, a badany algorytm \emph{Max. zysk + Analiza trafień} ma skuteczność ~50\% przeciwko algorytmom \emph{Max. zysk} oraz \emph{Max. zysk rozszerzony}. Wskazuje na to, że część algorytmu oparta na \emph{Analizie trafień} przestaje w tym scenariuszu przynosić korzyści względem pozostałych algorytmów.

Wpływ dodania statków 'średnich', tj. trzymasztowców przedstawiony jest w tabeli 7.12. Tym razem wyniki są bardziej zbliżone, ale znowu widać wzrost skuteczności badanego algorytmu \emph{Max. zysk + Analiza trafień} przeciwko \emph{Analizie trafień}, gdy statki nie mogą się ze sobą stykać. Tym razem jednak nie jest to spowodowane spadkiem efektywności heurystyki \emph{Analizy trafień}, a raczej wzrostem skuteczności \emph{Max. zysk}. Więcej statków oznacza, że w końcowej fazie rozgrywki, istnieje dużo mniej możliwych pozycji pozostałych statków - są one ograniczone pozycjami tych już zatopionych.

Tabela 7.13 przedstawia zmianę skuteczności \emph{Max. zysk + Analiza trafień} po wprowadzeniu dodatkowego cztero- i pięciomasztowca. Wpływ jest bardzo podobny jak w poprzednium przypadku, dla trzymasztowców. Heurystyka \emph{Max. zysk} staje się dużo bardziej pomocna. Ciekawy jest fakt, że pomimo dodanie 9 dodatkowych komórek do zestrzelenia, liczba tur w żadnym przypadku nie wzrasta o więcej niż 7, a przeciwko losowym algorytmom nawet maleje. To samo zjawisko widoczne jest też w tabeli 7.12. Wynika to z faktu, że dodatkowe statki oznaczają nie tylko dodatkowe cele, ale również dodatkowe ograniczenie pozycji pozostałych statków. Dla algorytmów opartych na heurystyce \emph{Max. zysk} większa liczba statków ułatwia wnioskowanie, a co za tym idzie zwiększa ich skuteczność.

Z tabel 7.14 i 7.15 wynika, że rozstawienie statków na planszy nie ma zbyt dużego wpływu na skuteczność algorytmów. W przypadku algorytmów losowych jest to spowodowane faktem, że strzały obu stron mają takie samo prawdopodobieństwo trafienia, niezależnie od tego jak rozstawione są statki. Algorytm \emph{Analiza trafień} jest podobny w tej kwestii - również oddaje losowe strzały, aż trafi przeciwnika. Algorytmy oparte na heurystyce \emph{Max. zysk} prawdopodobnie wystarczająco szybko eliminują potencjalne położenia statków na planszy i tym samym niwelują korzyści wynikające z danego rozstawienia statków.

\subsection{Ograniczenia pracy}
Nie udało się niestety zebrać wystarczająco dużo danych z rozgrywek człowiek-algorytm, aby reprezentatywnie ocenić skuteczność algorytmów w starciach z człowiekiem.
