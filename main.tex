%%%%%%%%%%%%%%%%%%%%%%%%%%%%%%%%%%%%%%%%%%%%%%%%%%%%%%%
%% Bachelor's & Master's Thesis Template             %%
%% Copyleft by Artur M. Brodzki & Piotr Woźniak      %%
%% Faculty of Electronics and Information Technology %%
%% Warsaw University of Technology, 2019-2020        %%
%%%%%%%%%%%%%%%%%%%%%%%%%%%%%%%%%%%%%%%%%%%%%%%%%%%%%%%

\documentclass[
    left=2.5cm,         % Sadly, generic margin parameter
    right=2.5cm,        % doesnt't work, as it is
    top=2.5cm,          % superseded by more specific
    bottom=3cm,         % left...bottom parameters.
    bindingoffset=6mm,  % Optional binding offset.
    nohyphenation=false % You may turn off hyphenation, if don't like.
]{eiti/eiti-thesis}

\langpol % Dla języka angielskiego mamy \langeng
\graphicspath{{img/}}             % Katalog z obrazkami.
\addbibresource{bibliografia.bib} % Plik .bib z bibliografią

% \usepackage[T1]{fontenc}
% \usepackage[scaled]{beramono}

\usepackage{color}
\definecolor{bluekeywords}{rgb}{0.13,0.13,1}
\definecolor{greencomments}{rgb}{0,0.5,0}
\definecolor{redstrings}{rgb}{0.9,0,0}

\usepackage[utf8]{inputenc}
\usepackage{listings}

\lstset{literate={↑}{$\uparrow$}1 {↓}{$\downarrow$}1 {←}{$\leftarrow$}1 {→}{$\rightarrow$}1}


\lstset{language=[Sharp]C,
showspaces=false,
showtabs=false,
breaklines=true,
showstringspaces=false,
breakatwhitespace=true,
escapeinside={(*@}{@*)},
commentstyle=\color{greencomments},
keywordstyle=\color{bluekeywords}\bfseries,
stringstyle=\color{redstrings},
breaklines=true
}

\lstdefinelanguage{JavaScript}{
  keywords={break, case, catch, continue, debugger, default, delete, do, else, finally, for, function, if, in, instanceof, new, return, switch, this, throw, try, typeof, var, void, while, with},
  morecomment=[l]{//},
  morecomment=[s]{/*}{*/},
  morestring=[b]',
  morestring=[b]",
  sensitive=true,
  breaklines=true
}

\newcommand{\tocless}[2]{\bgroup\let\addcontentsline=\nocontentsline#1{#2}\egroup}

\begin{document}

%--------------------------------------
% Strona tytułowa
%--------------------------------------
\MasterThesis % Dla pracy inżynierskiej mamy \EngineerThesis
\instytut{Automatyki i Informatyki Stosowanej}
\kierunek{Informatyka}
\specjalnosc{Systemy Internetowe Wspomagania Zarządzania}
\title{
    Implementacja gry strategicznej Statki, opracowanie algorytmów decyzyjnych oraz analiza ich skuteczności.
}
\engtitle{ % Tytuł po angielsku do angielskiego streszczenia
    Implementation of a Battleships strategy game, development of decision making algorithms and an analysis of their effectiveness.
}
\author{Jakub Kalinowski}
\album{284234}
\promotor{dr Sebastian Plamowski}
\date{\the\year}
\maketitle

%--------------------------------------
% Streszczenie po polsku
%--------------------------------------
\cleardoublepage % Zaczynamy od nieparzystej strony
\streszczenie{
Celem niniejszej pracy było zaimplementowanie gry strategicznej Statki i opracowanie algorytmów decyzyjnych, aby ostatecznie przeanalizować skuteczność poszczególnych algorytmów. Według wstępnych założeń, algorytmy miały opierać się na heurystykach. Analiza skuteczności miała opierać się na bezpośrednich starciach różnych algorytmów oraz na starciach człowiek - algorytm.
\\ \indent W pracy omówione zostały zagadnienia związane z algorytmami oraz heurystykami, z punktu widzenia logiki oraz informatyki. Opisane zostały typowe zastosowania heurystyk.
\\ \indent W dalszej części pracy opisane zostało zaimplementowane rozwiązanie - wykorzystany stack technologiczny, logika aplikacji, implementacja poszczególnych heurystyk. Opisane zostało działanie najważniejszych fragmentów kodu.
\\ \indent Kolejnym krokiem było testowanie algorytmów. W tym celu napisane zostały testy jednostkowe, które zestawiały przeciwko sobie poszczególne heurystyki. Na podstawie wyników tych starć dokonano analizy porównawczej. Podjęto również próbę analizy starć pomiędzy algorytmami oraz ludźmi.
\\ \indent W ostatniej części pracy podsumowane zostały wyniki oraz opisane zostały wnioski - co można było zrobić lepiej, jakie są możliwości rozwoju aplikacji.
}
\slowakluczowe heurystyki, statki, algorytmy, vue.js, .net

%--------------------------------------
% Streszczenie po angielsku
%--------------------------------------
\newpage
\abstract {
The purpose of this study was to implement the strategy game Battleships and to develop decision-making algorithms to analyze the effectiveness of the various algorithms. According to the initial assumptions, the algorithms were to be based on heuristics. The effectiveness analysis was to be based on direct clashes between different algorithms and on human-algorithm clashes.
\\ \indent The paper discusses issues related to algorithms and heuristics, from the point of view of logic and computer science. Typical applications of heuristics are described.
\\ \indent The following part of the work describes the implemented solution - the technology stack used, the application logic, the implementation of individual heuristics. The most important code fragments were highlighted and described.
\\ \indent The next step was to test the algorithms. For this purpose, unit tests were written that pitted individual heuristics against each other. Based on the results of these clashes, a comparative analysis was made.  An attempt was also made to analyze clashes between the algorithms and players.
\\ \indent In the last part of the work the results were summarized, and conclusions were written - what could have been done better, what are the possibilities for the development of the application.
}
\keywords heuristics, battleships, algorithms, vue.js, .net

%--------------------------------------
% Oświadczenie o autorstwie
%--------------------------------------
\cleardoublepage  % Zaczynamy od nieparzystej strony
\pagestyle{plain}
\makeauthorship

%--------------------------------------
% Spis treści
%--------------------------------------
\cleardoublepage % Zaczynamy od nieparzystej strony
\tableofcontents

%--------------------------------------
% Rozdziały
%--------------------------------------
\cleardoublepage % Zaczynamy od nieparzystej strony
\pagestyle{headings}

\newpage % Rozdziały zaczynamy od nowej strony.
\section{Wstęp}

\subsection{Wprowadzenie}
\indent\ Heurystyki, czyli uproszczone metody podejmowania decyzji, pomagają nam w codziennym życiu przy  podejmowaniu różnych decyzji, nawet jeśli nie jesteśmy tego do końca świadomi. Często korzystamy z heurystyk, aby szybciej podjąć decyzję - nie mamy wtedy jednak pewności że będzie to decyzja optymalna. "Heurystyka dostępności" jest często stosowana do szacowania prawdopodobieństwa zdarzeń, na podstawie tego jak łatwo jest nam przywołać przykłady wystąpienia tego zdarzenia - na przykład jeśli wielokrotnie słyszymy że na zaparkowany samochód spadło drzewo, będziemy unikać parkowania pod drzewem \cite{tversky74}.
\\ \indent Heurystyki w kontekście informatyki również ułatwiają rozwiązywanie problemów, których złożoność obliczeniowa jest na tyle duża, że ma to bardzo negatywny wpływ na czas wykonywania obliczeń. Jako przykład może służyć problem komiwojażera - dla podanych miast oraz dystansów między nimi, musimy wyznaczyć najkrótszą drogę wiodącą z miasta A do B, która przechodzi przez wszystkie inne miasta. Dla małej liczby miast, problem można rozwiązać metodą brute-force, obliczając wszystkie możliwe ścieżki i wybierając najkrótszą. Jednak z każdym dodanym do zadania miastem, znacznie wzrasta liczba możliwych ścieżek - a wraz z nią czas potrzebny na obliczenia. W tym celu Job Bentley zaproponował uproszczony algorytm, który nie bierze pod uwagę wszystkich ścieżek. Zamiast tego, dla każdego kolejnego kroku, wybiera po prostu najbliższe miasto. Ostatecznie wybrana ścieżka nie zawsze będzie faktycznie tą możliwie najkrótszą, ale będzie do niej zbliżona \cite{Hjeij2023}
\\ \indent Niniejsza praca zagłębia się w zastosowanie heurystyk w informatyce, a konkretnie w grach komputerowych. Jest do świetne narzędzie do implementacji algorytmów decyzyjnych, z których korzystają komputerowi przeciwnicy w grze. Szczególnie we wpółczesnych grach z otwartym światem, podejmowanie decyzji przez przeciwników nie może za bardzo obciążać zasobów procesora, ponieważ są one również potrzebne do obliczeń fizyki gry, symulacji otaczającego gracza świata, logiki gry, etc.
\\ \indent 

\subsection{Cel pracy}
\indent Celem niniejszej pracy była implementacja turowej gry strategicznej Statki oraz analiza skuteczności algorytmów decyzyjnych przeciwnika. W implementacji istotne było zapewnienie możliwości archiwizacji zakończonych rozgrywek w bazie danych, aby później możliwa była ich analiza.

\subsection{Metodologia}
\indent Przedmiotem badania była ocena skuteczności zaimplementowanych algorytmów. Na koniec pracy powinniśmy znać odpowiedzi na pytania:
\begin{enumerate}
  \item Który algorytm decyzyjny jest najbardziej skuteczny?
  \item Co ma wpływ na skuteczność algorytmów?
  \item Czy skuteczność algorytmów różni się w zależności od tego czy gra przeciwko człowiekowi, czy innemu algorytmowi?
\end{enumerate}
Aby opowiedzieć na te pytania, przeprowadzimy analizę statystyczną rozgrywek zapisanych w bazie danych.         % Wygodnie jest trzymać każdy rozdział w osobnym pliku.
\newpage % Rozdziały zaczynamy od nowej strony.
\section{Gra strategiczna Statki – podstawowe zasady i historia}

\subsection{Historia gry}
\indent\ Gra Statki, znana również jako "Battleships", to popularna gra strategiczna, której korzenie sięgają początku XX wieku. Jest to gra turowa dla dwóch graczy, którzy starają się zatopić flotę przeciwnika oddając na zmianę strzały na planszę przeciwnika. Gracze nie widzą planszy przeciwnika, po każdym oddanym strzale dostają jedynie informację zwrotną, czy trafili w statek.
\\ \indent Pierwsze wersje gry "Statki" były grami papierowymi, które były popularne wśród żołnierzy w trakcie I Wojny Światowej. Do gry potrzebne były jedynie 2 kartki oraz ołówek, więc można było w nią grać właściwie wszędzie. \cite{historyWiki}
\\ \indent Od lat 30 XX wieku, dostępne były komercyjne wersje gry - gracze mogli kupić gotowe papierowe plansze do gry, aby nie musieć własnoręcznie ich rysować. W 1967 firma Milton Bradley zaczęła produkować prawdopodobnie najbardziej znaną wersję gry, składającą się z plastikowych planszy oraz statków. \cite{historyWiki} \cite{museumOfGames}

\begin{figure}[!h]
    \label{fig:milton-bradley-game}
    \centering \includegraphics[width=0.8\linewidth]{img/milton_bradley_game.jpg}
    \caption{Gra Battleship wyprodukowana przez firmę Milton Bradley \cite{eBay}.}
\end{figure}

 \indent Z biegiem lat gra Statki ewoluowała, przyjmując różne formy. W latach 70. XX wieku pojawiła się pierwsze wersja elektroniczna, dostępna na komputerach Z80 Compucolor, napisana w języku BASIC. Wraz z rozwojem PC, zaczęło pojawiać się coraz więcej wersji gry.\cite{historyWiki} \cite{museumOfGames} .

 \begin{figure}[!h]
    \label{fig:commodore_64}
    \centering \includegraphics[width=0.8\linewidth]{img/commodore_64.jpg}
    \caption{Ujęcie z gry \emph{Battleship} dostępnej na komputerze Commodore 64 \cite{commodore64}.}
\end{figure}

\indent Gra Statki jest znanym elementem kultury popularnej na całym świecie, na jej podstawie w 2012 powstał nawet film "Battleship".
\subsection{Zasady}
\indent\ Zasady gry Statki różnią się w zależności od wersji. Zasady przyjęte w niniejszej pracy:
\begin{itemize}
  \item Gracze mają do dyspozycji po jednym statku o danej długości, zaczynając od jednomasztowca, kończąc na pięciomasztowcu.
  \item Dany statek musi leżeć w linii prostej (nie może na przykład być "złamany" i tworzyć literę L)
  \item Plansza ma wymiary 10 na 10.
  \item Gracz może wybrać czy statki mogą się ze sobą stykać bokami lub rogami. Ma to na celu dodanie dodatkowej zmiennej do analizy skuteczności algorytmów.
\end{itemize}

Różnorodność zasad to być może coś, co przyczyniło się do popularności tej gry - gracze mogą zmieniać zasady, tak aby rozgrywka była dla nich jak najciekawsza. Gra ograniczona wieloma zasadami, takimi jak to że statki nie mogą sąsiadować, prowadzi do bardziej analitycznej rozgrywki - gracze mogą eliminować z rozważań pola, na których zgodnie z zasadami nie mogą znajdować się statki. Gdy nie ma zasad ograniczających rozstawienie statków, gra robi się dużo bardziej losowa.    % Umożliwia to również łatwą migrację do nowej wersji szablonu:
\newpage % Rozdziały zaczynamy od nowej strony.
\section{Algorytmy}
\subsection{Wprowadzenie do algorytmów}

Algorytmy stanowią podstawę działania każdego programu komputerowego. Są to zestawy kroków, które prowadzą do rozwiązania określonego problemu. Każdy algorytm można opisać za pomocą języka formalnego, który jest zrozumiały dla komputera. W kontekście informatyki, algorytmy nie tylko rozwiązują problemy, ale także optymalizują procesy obliczeniowe, minimalizując czas i zasoby potrzebne do uzyskania wyniku\cite{algorithms}.

\subsection{Klasyfikacja algorytmów}

Algorytmy można klasyfikować według różnych kryteriów, w zależności od ich struktury, sposobu działania, oraz rodzaju problemów, które rozwiązują. Oto kilka głównych kategorii:

\begin{itemize}
    \item{\textbf{Algorytmy deterministyczne i niedeterministyczne}}
    
    Algorytmy deterministyczne zawsze prowadzą do tego samego wyniku, jeśli zostaną uruchomione z tymi samymi danymi wejściowymi. Przykładem jest algorytm sortowania bąbelkowego (bubble sort) [2]. Z kolei algorytmy niedeterministyczne mogą dawać różne wyniki przy tych samych danych wejściowych, jak np. algorytmy genetyczne \cite{cohen1979non}.
    
    \item{\textbf{Algorytmy dokładne i przybliżone}}
    
    Algorytmy dokładne znajdują optymalne rozwiązanie problemu, jednak często wymagają dużych zasobów obliczeniowych. Przykładem jest algorytm znajdowania najkrótszej ścieżki w grafie (algorytm Dijkstry). Algorytmy przybliżone, jak heurystyki, oferują rozwiązanie, które może nie być optymalne, ale jest uzyskane znacznie szybciej \cite{trevisan}.
    
    \item{\textbf{Algorytmy iteracyjne i rekurencyjne}}

    Algorytmy iteracyjne rozwiązują problem poprzez powtarzanie określonych operacji, aż do osiągnięcia pożądanego rezultatu. Algorytmy rekurencyjne natomiast rozwiązują problem, wywołując samą siebie na mniejszych podproblemach. Przykładem algorytmu rekurencyjnego jest algorytm wyszukiwania binarnego \cite{recursiveVsIterative}.
    
\end{itemize}

\subsection{Przykłady popularnych algorytmów}
\begin{itemize}
    \item{\textbf{Algorytmy sortowania}}

    Sortowanie to proces porządkowania elementów w określonej kolejności. Istnieje wiele algorytmów sortowania, takich jak quicksort, mergesort, czy bubble sort. Każdy z nich ma swoje wady i zalety w zależności od danych wejściowych i kontekstu, w którym jest stosowany \cite{algorithms}.
    
    \item{\textbf{Algorytmy wyszukiwania}}

    Algorytmy wyszukiwania służą do znalezienia określonego elementu w strukturze danych. Przykładami są algorytm wyszukiwania liniowego oraz algorytm wyszukiwania binarnego, który działa znacznie szybciej w posortowanych zbiorach danych \cite{sedgewick}.
    
    \item{\textbf{Algorytmy grafowe}}

    Algorytmy działające na grafach są kluczowe w wielu dziedzinach, od analizy sieci po optymalizację tras. Do najważniejszych należą algorytm Dijkstry do znajdowania najkrótszej ścieżki, algorytm Kruskala i algorytm Prima do znajdowania minimalnego drzewa rozpinającego \cite{algorithms}.
    
    \item{\textbf{Algorytmy sztucznej inteligencji}}

    Wraz z rozwojem sztucznej inteligencji, algorytmy AI zyskały na znaczeniu. Algorytmy te, takie jak algorytmy genetyczne, sieci neuronowe czy algorytmy uczące się, umożliwiają komputerom naśladowanie procesów myślowych człowieka, podejmowanie decyzji oraz uczenie się na podstawie danych \cite{Mitchell1996AnIT}.
    
\end{itemize}

\subsection{Złożoność obliczeniowa algorytmów}

Złożoność obliczeniowa odnosi się do ilości zasobów, takich jak czas i pamięć, które są potrzebne do wykonania algorytmu. Jest to kluczowy aspekt przy ocenie efektywności algorytmu. Złożoność obliczeniową wyraża się zazwyczaj w notacji O (Big O), która opisuje asymptotyczne zachowanie algorytmu, czyli jak rośnie jego czas wykonywania wraz ze wzrostem wielkości danych wejściowych \cite{artoOfProgramming}.

Przykłady:
\begin{itemize}
    \item{\textbf{O(1)}}

    Stała złożoność czasowa, niezależna od wielkości danych wejściowych.
    
    \item{\textbf{O(n)}}

    Złożoność liniowa, czas wykonywania rośnie proporcjonalnie do wielkości danych wejściowych.
    
    \item{\textbf{O(n^{2})}}

    Złożoność kwadratowa, czas wykonywania rośnie kwadratowo wraz ze wzrostem danych wejściowych.
    
\end{itemize}

\subsection{Znaczenie algorytmów w informatyce} Współczesna informatyka nie mogłaby istnieć bez algorytmów. Stanowią one fundament każdej aplikacji, od prostych programów po zaawansowane systemy komercyjne. Znajomość algorytmów i umiejętność ich optymalizacji jest kluczowa dla programistów, inżynierów oprogramowania oraz naukowców zajmujących się sztuczną inteligencją, big data czy analizą danych.
\newpage % Rozdziały zaczynamy od nowej strony.
\section{Heurystyki}

\subsection{Definicja i rola heurystyk}
\indent\ Termin "heurystyka" pochodzi od greckiego \emph{heurískō}, co oznacza \emph{znajdować/odkrywać} \cite{heuristicEtymology}. Heurystyki to uproszczone metody podejmowania decyzji, które umożliwiają szybkie i efektywne rozwiązywanie problemów.

Badania nad heurystykami zostały rozpoczęte przez psychologów Amosa Tversky'ego oraz Daniela Kahnemana w latach 70 XX wieku. Opisane przez nich heurystyki stanowiły wytłumaczenie dla systematycznych odstępstw od teoretycznie bardziej racjonalnych decyzji. Podczas podejmowania decyzji ludzie stosują skróty myślowe, które ułatwiają im podjęcie decyzji, ale jednocześnie sprawiają, że są bardziej podatni na błędy poznawcze (ang. cognitive bias). Tversky i Kahneman sformułowali wiele heurystyk używanych w psychologii do dziś, między innymi heurystyki dostępności, zakotwiczenia i dostosowania czy reprezentatywności \cite{tversky74} \cite{psychol} \cite{laibson}.

Przykłady heurystyk:
\begin{itemize}
    \item{\textbf{Dostępności}}: Oceniamy prawdopodobieństwo zdarzenia, na podstawie tego jak łatwo możemy przywołać odpowiednie przykłady. Jeśli jakieś zdarzenie lub informacja jest bardziej dostępna w naszej pamięci, mamy tendencję do przeceniania jej prawdopodobieństwa wystąpienia \cite{tversky74}.
    \item{\textbf{Zakotwiczenia i dostosowania}}: Przy podejmowaniu decyzji polegamy na początkowej informacji (zakotwiczeniu), ale nie dostosowujemy wagi tej informacji po otrzymaniu nowych danych. Przykładem może być sytuacja, gdy konsument początkowo widzi wysoką cenę produktu. Jeśli następnie zobaczy obniżkę, pomyśli że jest to bardzo korzystna oferta. W przeciwnej sytuacji, jeśli najpierw zobaczy przeceniony produkt, a następnie w normalnym wariancie cenowym, będzie wydawał mu się zbyt drogi \cite{tversky74, pricing}.
    \item{\textbf{Reprezentatywności}}: Na podstawie stereotypu, oceniamy prawdopodobieństwo przynależności osoby do danej grupy. Przykładowo: jeśli widzimy zgarbionego, bladego chłopaka w okularach i zostaniemy zapytani czy bardziej prawdopodobne jest to czy jest informatykiem czy rolnikiem, wybierzemy pierwszą opcję. Z logicznego punktu widzenia nie jest to dobry wybór, ponieważ na świecie jest więcej rolników niż informatyków. Dzieje się tak, ponieważ wymienione cechy pasują do istniejącego w naszej głowie stereotypu \cite{tversky74} \cite{sudeep}.
\end{itemize}

\subsection{Heurystyki w grach strategicznych}
W kontekście gier strategicznych, heurystyki są kluczowe dla efektywnego zarządzania zasobami, planowania i adaptacji strategii. Komputerowi przeciwnicy korzystają z heurystyk, aby naśladować ludzkie myślenie i podejmowanie decyzji, co czyni rozgrywkę bardziej wymagającą i interesującą dla graczy.

Przykładowe heurystyki wykorzystywane w grach:
\begin{itemize}
  \item \textbf{Heurystyki oparte na odległości:} Przykładem jest heurystyka najkrótszej ścieżki, która służy do oceny odległości między jednostkami lub celami w grze, co pomaga w planowaniu ruchów i ataków \cite{inproceedings}.
  \item \textbf{Heurystyki oparte na priorytetach:} Pozwalają graczom ustalać priorytety dla różnych celów lub działań. Na przykład, w grach RTS (Real-Time Strategy), przeciwnik może priorytetowo traktować budowanie zasobów lub atakowanie gracza w początkowej fazie gry \cite{Ontan2013ASO}.
  \item \textbf{Heurystyki oparte na doświadczeniu:} Korzystają z wcześniej zdobytych doświadczeń lub wyników gier do podejmowania decyzji. Przykładem może być rozpoznawanie wzorców w ruchach gracza i dostosowywanie strategii na ich podstawie \cite{Weber_Mateas_Jhala_2011}.
\end{itemize}

\subsection{Heurystyki zastosowane w pracy}
\begin{enumerate}
  \item \textbf{Heurystyka maksymalizacji zysku ze strzału}
  
   Przeciwnik analizuje ile istnieje możliwości ustawienia statków dla danej komórki. Na rysunku 4.1 widoczna jest mapa prawdopodobieństwa, obliczona na podstawie tej heurystyki. Wartości widoczne na każdej komórce oznaczają na ile sposobów można ustawić dostępne statki na danej komórce. Pola będące krawędziami planszy mają wartość 10, ponieważ może na nich znaleźć się każdy ze statków w pionie i w poziomie, w obu przypadkach w dokładnie jednej pozycji. Pola znajduujące się bliżej środka mapy mają większe wartości, ponieważ statki mogą znajdować się na nich w wielu pozycjach. Na przykład poziomy trzymasztowiec może znajdować się na polu X, Y na 3 sposoby. Poniżej wymienione są jego możliwe współrzędne:
  \begin{itemize}
      \item (X-2,Y), (X-1,Y), (X,Y)
      \item (X-1,Y), (X,Y), (X+1,Y)
      \item (X,Y), (X+1,Y), (X+2,Y)
  \end{itemize}

  Heurystyka maksymalizacji zysku ze strzału pozwala na wykonywanie jak najbardziej efektywnych strzałów pod kątem eliminacji możliwych rozstawień statków.
  
  
  \begin{figure}[!h]
    \label{fig:mapa-prawdopodobienstwa-heurystyka-max-zysku}
    \centering \includegraphics[width=0.5\linewidth]{img/probabilityMapStart.PNG}
    \caption{Mapa prawdopodobieństwa podczas rozpoczęcia gry, przy wykorzystaniu heurystyki maksymalizacji zysku ze strzału.}
\end{figure}
  
  \item \textbf{Heurystyka najbardziej prawdopodobnej lokalizacji na podstawie trafień}

  
  Przeciwnik analizuje swoje dotychczasowe trafienia i skupia się na ostrzale komórek sąsiadujących z pojedynczymi trafieniami oraz wzdłuż linii kilku sąsiadujących trafień. Zatopione statki nie są brane pod uwagę. Jeśli żaden statek nie został jeszcze trafiony, strzały oddawane są losowo. Na rysunku 4.2 widoczne są dwie mapy prawdopodobieństwa. Piersza mapa przedstawia sytuację, gdzie trafiona została pojedyncza komórka - wtedy wszystkie sąsiadujące komórki mogą zawierać pozostałą część statku. Na drugim rysunku trafione zostały dwie sąsiadujące komórki - wtedy rozważane są jedynie komórki sąsiadujące, będące na przedłużeniu linii dotychczasowych trafień. Wartości przedstawione na mapie prawdopodobieństwa są umówne - 50 dla komórek sąsiadujących z pojedynczym trafieniem oraz 100 dla komórek sąsiadujących z serią trafień.
  \begin{figure}[!h]
    \label{fig:mapa-prawdopodobienstwa-heurystyka-trafien}
    \centering \includegraphics[width=0.9\linewidth]{img/hit-heuristic.png}
    \caption{Mapy prawdopodobieństwa dla dwóch stanów planszy gracza, przy wykorzystaniu heurystyki najbardziej prawdopodobnej lokalizacji na podstawie trafień.}
\end{figure}
  
  \item \textbf{Heurystyka maksymalizacji zysku priorytetyzująca dłuższe statki}
  
  Bardzo podobna do już wspomnianej heurystyki maksymalizacji zysku, z jedną różnicą - priorytetyzowane są dłuższe statki. W podstawowej wersji heurystyki, jeśli statek mógł znajdować się na komórce (X,Y) w liczbie Z pozycji, dodawaliśmy Z do tej komórki na mapie prawdopodobieństwa. Teraz dodajemy
  \begin{align*}
    Z * L^2
\end{align*}
Gdzie L to długość statku.
  \\ Heurystyka ta pozwala na szybszą eliminację dużych statków.
    \begin{figure}[!h]
    \label{fig:mapa-prawdopodobienstwa-max-zysku-wazona}
    \centering \includegraphics[width=0.5\linewidth]{img/max-benefit-weighted.PNG}
    \caption{Mapa prawdopodobieństwa podczas rozpoczęcia gry, przy wykorzystaniu heurystyki maksymalizacji zysku priorytetyzującej dłuższe statki.}
    \end{figure}

  \item \textbf{Heurystyka maksymalizacji zysku oraz najbardziej prawdopodobnej lokalizacji na podstawie trafień}
  
  Połączenie heurystyki maksymalizacji zysku ze strzału (1) oraz heurystyki najbardziej prawdopodobnej lokalizacji na podstawie trafień (2). Ulepszenie podstawowej heurystyki najbardziej prawdopodobnej lokalizacji na podstawie trafień - zamiast losowo ostrzeliwać pola, gdy żadne nie jest jeszcze trafione, teraz strzały oddawane są w pola, które wskazuje heurystyka maksymalizacji zysku. Wagi obu heurystyk są tak dopasowane, aby heurystyka dotychczasowych trafień miała większy wpływ na wybór komórki do ostrzału.

    \begin{figure}[!h]
    \label{fig:mapa-prawdopodobienstwa-heurystyka-laczona}
    \centering \includegraphics[width=0.9\linewidth]{img/complete-heuristic.png}
    \caption{Mapa prawdopodobieństwa podczas rozpoczęcia gry, przy wykorzystaniu heurystyki maksymalizacji zysku oraz najbardziej prawdopodobnej lokalizacji na podstawie trafień.}
    \end{figure}

  \item \textbf{Heurystyka maksymalizacji zysku oraz najbardziej prawdopodobnej lokalizacji na podstawie trafień priorytetyzująca dłuższe statki}
  
  Połączenie heurystyki maksymalizacji zysku ze strzału priorytetyzującej dłuższe statki (3) oraz heurystyki najbardziej prawdopodobnej lokalizacji na podstawie trafień (2). Wagi obu heurystyk są tak dopasowane, aby heurystyka dotychczasowych trafień miała większy wpływ na wybór komórki do ostrzału. W sytuacji, gdy żadne pole nie jest jeszcze trafione, priorytetyzowane są komórki, które mogą być położeniem najdłuższych statków.
\end{enumerate}
\input{tex/5-implementacja}
\input{tex/6-przeglad}
\newpage % Rozdziały zaczynamy od nowej strony.
\section{Analiza skuteczności algorytmów}

\subsection{Metody testowania}
\indent\ Testy zostały przeprowadzone za pomocą testów jednostkowych, w których wszystkie algorytmy zostały zestawione przeciwko sobie. Statki były rozstawiane losowo dla obu stron, z wyjątkiem testów w których analizowany był wpływ rozstawienia statków na skuteczność. Wszystkie testy jednostkowe dziedziczą po klasie bazowej \emph{AlgorithmTestBase}, która była potrzebna do inicjalizacji lub zmockowania wszelkich potrzebnych serwisów. Na listingu 12 widoczny jest przykładowy test jednostkowy. Testowane były oba warianty zasady stykania się statków. Stała \emph{numberOfIterations} wynosiła 1000, więc dla każdego zestawienia algorytmów i zasad otrzymano 1000 wyników. Ogólnie podczas testów zebrano około 240 000 rekordów w kontenerze \emph{TestGameSessions}.

\begin{addmargin}[10mm]{0mm}
\begin{lstlisting}[
    language={[Sharp]C},
    numbers=left,
    firstnumber=11,
    caption={Przykładowy test jednostkowy},
    aboveskip=0pt
]
[Fact]
public void VsRandomShipsCantTouch()
{
    _httpContextAccessor.SetupTestHttpContext();

    var initialGameState = new GameState
    {
        PlayerAiType = AiType.LocationHeuristic,
        OpponentAiType = AiType.Random,
        ShipsCanTouch = false,
    };

    var playerWins = TestHelper.RunSimulation(
        initialGameState,
        _gameStateService,
        _generateMoveService,
        _shipLocationService,
        numberOfIterations);

    Assert.True(true);
}
\end{lstlisting}
\end{addmargin}


\subsection{Badanie skuteczności algorytmów przeciwko sobie}
W tabelach widocznych w poniższych rozdziałach stosowane będą skrócone nazwy algorytmów, tak aby bez problemu mieściły się w komórkach tabel. Poniżej znajduje się objaśnienie skrótów.
\begin{itemize}
    \item \textbf{Max. zysk} - Algorytm oparty na heurystyce maksymalizacji zysku ze strzału.
    \item \textbf{Analiza trafień} - Algorytm oparty na heurystyce najbardziej prawdopodobnej lokalizacji na podstawie trafień.
    \item \textbf{Max. zysk rozszerzony} - Algorytm oparty na heurystyce maksymalizacji zysku priorytetyzującej dłuższe statki.
    \item \textbf{Max. zysk + analiza trafień} - Algorytm oparty na heurystykach maksymalizacji zysku oraz najbardziej prawdopodobnej lokalizacji na podstawie trafień.
    \item \textbf{Max. zysk rozszerzony + analiza trafień} - Algorytm oparty na heurystykach maksymalizacji zysku oraz najbardziej prawdopodobnej lokalizacji na podstawie trafień priorytetyzującej dłuższe statki. 
\end{itemize}

W tabelach przedstawiających skuteczność poszczególnych algorytmów posłużono się kolorami, w celu podkreślenia otrzymanych wartości:
\begin{itemize}
    \item Kolor zielony - wynik zdecydowanie lepszy, skuteczność powyżej 55\%
    \item Kolor żółty - wynik porównywalny, skuteczność pomiędzy 45\%-55\%
    \item Kolor czerwony - wynik zdecydowanie gorszy, skuteczność poniżej 45\%
\end{itemize}

\subsubsection{Algorytm losowy}

Algorytm losowy, zgodnie z przewidywaniami wypadł najgorzej w zestawieniu z innymi algorytmami. Z tabeli 7.1 wynika, że najwyższy procent zwycięstw jaki osiągnął to 13,1\% przeciwko algorytmowi losowemu rozszerzonemu. Wśród wyników można zaobserwować prawidłowość, że skuteczność delikatnie wzrasta w przypadku testów, gdzie statki mogą się stykać. Zgodnie z tym, co opisano pod koniec rozdziału 2.2, dodatkowe ograniczenie tego, jak mogą być rozstawione statki, ułatwia heurystykom analizę planszy przeciwnika. Gdy tego ograniczenia nie ma, gra staje się trochę bardziej losowa - chociaż nadal algorytmy oparte na heurystykach są znacznie skuteczniejsze od algorytmu losowego.

\begin{table}[!h]
    \centering
    \includegraphics[width=1\linewidth]{img/table-random.png}
    \caption{Wyniki testów dla algorytmu losowego}
\end{table}

\subsubsection{Algorytm losowy z pominięciem pól sąsiadujących z zatopionymi statkami}

Algorytm losowy z pominięciem pól był testowany jedynie w przypadku gdy statki nie mogą się ze sobą stykać. W przeciwnym wypadku, gdyby oponent ustawił statki obok siebie, badany algorytm nie byłby w stanie odnieść zwycięstwa. W wynikach testów widocznych w tabeli 7.2 widać, że algorytm rozszerzony jest dużo skuteczniejszy od swojej podstawowej wersji. Można też zauważyć zdecydowany wzrost skuteczności przeciwko algorytmowi opartemu na heurystyce najbardziej prawdopodobnej lokalizacji na podstawie trafień. W pozostałych przypadkach również można zaobserwować wzrost skuteczności, ale jest on bardzo niewielki.

\begin{table}[!h]
    \centering
    \includegraphics[width=1\linewidth]{img/table-random-plus.png}
    \caption{Wyniki testów dla algorytmu losowego rozszerzonego}
\end{table}

\subsubsection{Algorytm oparty na heurystyce najbardziej prawdopodobnej lokalizacji na podstawie trafień}

Z tabeli 7.3 wynika, że w przypadku tego algorytmu widoczna jest wyraźnie jego przewaga nad algorytmami losowymi. Jest on jednak zdecydowanie najsłabszym z algorytmów heurystyczych. Umiejętność 'dobijania' trafionych statków wyróżnia go na tle algorytmów losowych, ale nadal w dużej mierze opiera się na losowości - jeśli na planszy przeciwnika nie ma żadnego trafienia, algorytm oddaje strzały w losowe komórki.

Podobnie jak w 7.2.1, widać znaczącą różnicę w zależności od wyboru zasad - skuteczność tego algorytmu wzrasta gdy statki mogą się ze sobą stykać.

\begin{table}[!h]
    \centering
    \includegraphics[width=1\linewidth]{img/table-hit-heuristic.png}
    \caption{Wyniki testów dla algorytmu opartego na heurystyce najbardziej prawdopodobnej lokalizacji na podstawie trafień}
\end{table}

\subsubsection{Algorytm oparty na heurystyce maksymalizacji zysku ze strzału}

W przypadku tego algorytmu widać wzrost skuteczności o około 10\% względem algorytmów losowych. Algorytm ten jest też zdecydowanie skuteczniejszy od algorytmu opartego na heurystyce najbardziej prawdopodobnej lokalizacji na podstawie trafień, aczkolwiek gdy statki mogą się ze sobą stykać, to ich skuteczność jest porównywalna.

\begin{table}[!h]
    \centering
    \includegraphics[width=1\linewidth]{img/table-location-heuristic.png}
    \caption{Wyniki testów dla algorytmu opartego na heurystyce maksymalizacji zysku ze strzału}
\end{table}


\subsubsection{Algorytm oparty na heurystyce maksymalizacji zysku priorytetyzującej dłuższe statki}

Wyniki zaprezentowane w tabeli 7.5 są bardzo zbliżone do wartości z punktu 7.2.4. Można zaobserwować, że w bezpośrednim starciu, badany algorytm jest nieznacznie skuteczniejszy od swojego podstawowego wariantu. Jednak we wszystkich pozostałych przypadkach, poza jednym, widoczny jest nieznaczny spadek skuteczności, w najgorszym przypadku 3,5\%.

\begin{table}[!h]
    \centering
    \includegraphics[width=1\linewidth]{img/table-location-heuristic-extended.png}
    \caption{Wyniki testów dla algorytmu opartego na heurystyce maksymalizacji zysku priorytetyzującej dłuższe statki}
\end{table}

\subsubsection{Algorytm oparty na heurystykach maksymalizacji zysku oraz najbardziej prawdopodobnej lokalizacji na podstawie trafień}

Jak wynika z danych w tabeli 7.6, algorytm wygrywa w zestawieniu ze wszystkimi pozostałymi algorytmami, poza swoją rozszerzoną wersją. Największy wzrost skuteczności widać przeciwko algorytmom \emph{Analiza trafień} oraz \emph{Max.zysk}. Wynika to z faktu, że badany algorytm jest usprawnieniem \emph{Max.zysk} poprzez umożliwienie algorytmowi 'dobijania' trafionych statków.

\begin{table}[!h]
    \centering
    \includegraphics[width=1\linewidth]{img/table-location-hit-heuristic.png}
    \caption{Wyniki testów dla algorytmu opartego na heurystykach maksymalizacji zysku oraz najbardziej prawdopodobnej lokalizacji na podstawie trafień}
\end{table}

\subsubsection{Algorytm oparty na heurystykach maksymalizacji zysku oraz najbardziej prawdopodobnej lokalizacji na podstawie trafień priorytetyzującej dłuższe statki}

Podobnie jak w przypadku 7.2.5, wyniki wersji podstawowej i rozszerzonej algorytmu są bardzo zbliżone. W bezpośrednim starciu wersja rozszerzona jest nieznacznie skuteczniejsza, gdy statki nie mogą się ze sobą stykać. W przeciwnym wypadku, lepiej wypada wersja podstawowa algorytmu. W starciach z pozostałymi algorytmami wyniki są zbliżone.

\begin{table}[!h]
    \centering
    \includegraphics[width=1\linewidth]{img/table-location-extended-hit-heuristic.png}
    \caption{Wyniki testów dla algorytmu opartego na heurystykach maksymalizacji zysku oraz najbardziej prawdopodobnej lokalizacji na podstawie trafień priorytetyzującej dłuższe statki}
\end{table}

\subsection{Podsumowanie}

Aby wyłonić najskuteczniejszy algorytm, utworzono tabele 7.8 oraz 7.9. Przedstawiają one zestawienie pojedynków pomiędzy wszystkimi algorytmami. Tabela 7.8 przedstawia wyniki, gdy statki nie mogą się ze sobą stykać, a tabela 7.9. przeciwny przypadek. W każdej kolumnie na zielono zaznaczone zostały komórki zawierające najwyższą skuteczność. Oznacza to, że algorytm przypisany do danego wiersza, był najskuteczniejszy w starciu z algorytmem przypisanym do danej kolumny.

Analizując dane tabel 7.8 i 7.9, można zaobserwować, że:

\begin{itemize}
    \item Gdy statki nie mogą się ze sobą stykać, podstawowa wersja  \emph{Max. zysk + analiza trafień} jest najskuteczniejsza w prawie wszystkich przypadkach. 
    \item Gdy statki mogą się ze sobą stykać, rozszerzona wersja  \emph{Max. zysk + analiza trafień} jest najskuteczniejsza w prawie wszystkich przypadkach, poza bezpośrednim starciem ze swoją podstawową wersją.
\end{itemize}

Widać więc wyraźnie wpływ różnicy zasad na wyniki. Ciekawe jest to, że w obu przypadkach ogólnie najskuteczniejszy algorytm przegrywał w starciu ze swoim drugim wariantem.

Aby ostatecznie rozstrzygnąć, który algorytm jest bardziej skuteczny spróbowano również policzyć ich ogólną skuteczność. Dokonano tego poprzez obliczenie średniej ze skuteczności przeciwko poszczególnym algorytmom. Przekłada się to na następujące wyniki procentowe:

\begin{itemize}
    \item 70,94\% dla wersji podstawowej
    \item 70,51\% dla wersji rozszerzonej
\end{itemize}

Nieznacznie skuteczniejszy jest więc wersja podstawowa \emph{Max. zysk + analiza trafień}.
\begin{table}[!h]
    \centering
    \includegraphics[width=1\linewidth]{img/summary-ships-cant-touch.PNG}
    \caption{Podsumowanie testów gdy statki nie mogą się ze sobą stykać}
\end{table}

\begin{table}[!h]
    \centering
    \includegraphics[width=1\linewidth]{img/summary-ships-can-touch.PNG}
    \caption{Podsumowanie testów gdy statki mogą się ze sobą stykać}
\end{table}

Analogicznie obliczono zagregowane skuteczności dla wszystkich algorytmów, co przedstawia tabela 7.10. Przy obliczeniach wykorzystano zapytanie widoczne na listingu 13.

\begin{table}[!h]
    \centering
    \includegraphics[width=0.7\linewidth]{img/aggregate.PNG}
    \caption{Zagregowane średnie skuteczności poszczególnych algorytmów}
\end{table}

\begin{addmargin}[10mm]{0mm}
\begin{lstlisting}[
    language=SQL,
    numbers=left,
    firstnumber=1,
    caption={Zapytanie do Azure Cosmos DB, w celu uzyskania zagregowanej skuteczności algorytmu},
    aboveskip=0pt
]
SELECT 
    AVG(
        IIF(c.playerMovesCount > c.opponentMovesCount,
        c.playerMovesCount,
        c.opponentMovesCount)
    ) AS avgMaxMoves
FROM c
WHERE (c.playerAiType = {ALG_1} AND c.opponentAiType = {ALG_2}
OR c.playerAiType = {ALG_2} AND c.opponentAiType = {ALG_1})
and c.shipsCanTouch = {WYBRANE_ZASADY}
\end{lstlisting}
\end{addmargin}

Na podstawie wyników stworzono wykres widoczny na rysunku 7.1. Algorytmy zostały posortowane od najmniej do najbardziej skutecznego.

\begin{figure}[!h]
    \label{fig:aggregate-chart}
    \centering \includegraphics[width=0.9\linewidth]{img/aggregate-chart.png}
    \caption{Wykres zagregowanych średnich dla poszczególnych algorytmów.}
\end{figure}

Utworzono również wykres 7.2 obrazujący skuteczności różnych algorytmów w starciach z algorytmem losowym, aby lepiej zobrazować różnice w skuteczności, zależnie od tego czy statki mogą się ze sobą stykać, czy nie.

\begin{figure}[!h]
    \label{fig:round-avg}
    \centering \includegraphics[width=0.9\linewidth]{img/chart-random-scores.png}
    \caption{Skuteczność algorytmów w starciach z algorytmem losowym.}
\end{figure}

Dla wszystkich zestawień algorytmów obliczono również średnią liczbę zagranych tur, zanim jedna ze stron odniosła zwycięstwo. Dane te widoczne są w tabelach 7.1-7.9 oraz na wykresie z rysunku 7.3, gdzie jako punkt odniesienia wykorzystano algorytm losowy.

\begin{figure}[!h]
    \label{fig:round-avg}
    \centering \includegraphics[width=0.9\linewidth]{img/round-count-avg.png}
    \caption{Średnia liczba tur w starciach z algorytmem losowym.}
\end{figure}

\subsection{Badanie wpływu liczby statków na skuteczność algorytmów}
Aby zbadać wpływ liczby statków na skuteczność algorytmów, powtórzone testy najbardziej skutecznego algorytmu \emph{Max. zysk + Analiza trafień}, w 3 różnych wariantach:
\begin{itemize}
    \item Obie strony mają do dyspozycji dodatkowe 2 jednomasztowce.
    \item Obie strony mają do dyspozycji dodatkowy pięciomasztowiec oraz czteromasztowiec.
    \item Obie strony mają do dyspozycji dodatkowe 2 trzymasztowce.
\end{itemize}

W tabelach w kolejnych podrozdziałach komórki tabel oznaczane są na czerwono, jeśli spadek skuteczności przekracza 5\%, na zielono jeśli wzrost jest większy od 5\% oraz na żółto jeśli zmiana skuteczności mieści się pomiędzy -5\% i 5\%.

\subsubsection{Dodatkowe najmniejsze statki}

W tabeli 7.11 widać wyraźny spadek skuteczności \emph{Max. zysk + Analiza trafień} w większości przypadków poza \emph{Analizą trafień}. Liczba tur wymaganych do zwycięstwa wzrasta, co było łatwe do przewidzenia, jako że obie strony mają teraz więcej statków do zestrzelenia.

\begin{table}[!h]
    \label{fig:vs-people}
    \centering \includegraphics[width=0.9\linewidth]{img/shipCountSmallShips.png}
    \caption{Skuteczność \emph{Max. zysk + Analiza trafień} przy dodatkowych jednomasztowcach.}
\end{table}

\subsubsection{Dodatkowe średnie statki}

W tabeli 7.12 widzimy, że skuteczność algorytmu jest bardzo zbliżona do skuteczności przy 'normalnej' liczbie statków. Po raz kolejny widoczny jest wzrost skuteczności przeciwko \emph{Analizie trafień}.

\begin{table}[!h]
    \label{fig:vs-people}
    \centering \includegraphics[width=0.9\linewidth]{img/shipCountMiddleShips.png}
    \caption{Skuteczność \emph{Max. zysk + Analiza trafień} przy dodatkowych trzymasztowcach.}
\end{table}

\subsubsection{Dodatkowe największe statki}

Podobnie do poprzedniego podpunktu, w tabeli 7.13 widać, że skuteczność \emph{Max. zysk i analiza trafień} jest zbliżona do  skuteczności przy 'normalnej' liczby statków. Znowu widać wzrost skuteczności przeciwko \emph{Analizie trafień}. Co ciekawe, w kilku przypadkach liczba tur jest mniejsza niż w przy 'normalnej' liczbie statków.

\begin{table}[!h]
    \label{fig:vs-people}
    \centering \includegraphics[width=0.9\linewidth]{img/shipCountBigShips.png}
    \caption{Skuteczność \emph{Max. zysk + Analiza trafień} przy dodatkowych cztero- i pięciomasztowcu.}
\end{table}

\subsection{Badanie wpływu rozstawienia statków na skuteczność algorytmów}
Postanowiono zbadać również wpływ rozstawienia statków na skuteczność algorytmów. W tym celu przeprowadzono analizę starć pomiędzy identycznymi algorytmami w 4 przypadkach:

\begin{itemize}
    \item Gdy statki są rozstawiane losowo przez obie strony.
    \item Gdy jedna ze stron częściej ustawia statki w narożnikach planszy.
    \item Gdy jedna ze stron częściej ustawia statki po lewej stronie planszy.
    \item Gdy jedna ze stron częściej ustawia statki w centrum planszy.
\end{itemize}

W tym celu zmodyfikowano kod odpowiadający za losowe rozstawianie statków, tak aby dla każdego statku istniało 80\% szans, że zostanie ustawiony w preferowanym obszarze planszy.

Korzystając ze skryptów \emph{GetShipLocations.py} oraz \emph{getData.py}, dostępnych w załączniku 3, wygenerowane zostały mapy cieplne widoczne na rysunku 7.4. Przedstawiają one to jak często dana komórka była częścią statku - im ciemniejszy kolor, tym częściej.

\begin{figure}[!h]
    \label{fig:bias-benchmark}
    \centering \includegraphics[width=1\linewidth]{img/biases.jpg}
    \caption{Mapy cieplne statków dla różnych tendencji przy rozstawianiu.}
\end{figure}

Na podstawie wyników rozgrywek sporządzono tabele 7.14 i 7.15. Każdy z wierszy opisuje skuteczność danego algorytmu w starciu z identycznym algorytmem, ale wykorzystującym różne tendencje przy rozstawianiu statków. Dla przykładu - komórka odpowiadająca parze "Losowy" - "Narożniki", opisuje skuteczność algorytmu losowego w procentach, przeciwko algorytmowi losowemu, który częściej rozstawia swoje statki w narożnikach planszy. Wszystkie komórki oznaczone są kolorem żółtym, ponieważ nie różnią się o więcej niż 5\% od punktu odniesienia - czyli skuteczności danego algorytmu gdy obie strony losowo rozstawiają statki.

\begin{table}[!h]
    \label{fig:biases-cant-touch}
    \centering \includegraphics[width=0.7\linewidth]{img/bias_result_cant_touch.png}
    \caption{Skuteczność algorytmów przeciwko sobie dla różnych tendencji rozstawienia statków, gdy statki nie mogą się stykać.}
\end{table}

W tabeli 7.15, podobnie jak w poprzednim przypadku, również nie widać wpływu rozstawienia statków na skuteczność algorytmów.

\begin{table}[!h]
    \label{fig:biases-can-touch}
    \centering \includegraphics[width=0.7\linewidth]{img/bias_result_can_touch.png}
    \caption{Skuteczność algorytmów przeciwko sobie dla różnych tendencji rozstawienia statków, gdy statki mogą się stykać.}
\end{table}

\subsection{Analiza rozgrywek z ludźmi}

Niestety nie udało się zebrać wystarczająco dużo danych, aby przeprowadzić dokładną analizę skuteczności algorytmów decyzyjnych przeciwko ludziom. Testy algorytm-algorytm opierały się na 1000 rozgrywek pomiędzy parami algorytmów, dla obu wariantów zasad. W przypadku testów z ludźmi, do analizy potrzebnych byłoby 13 000 rozgrywek, ponieważ zaimplementowano 7 algorytmów, w tym jeden który dostępny jest jedynie gdy statki nie mogą się ze sobą stykać. Udało się zebrać dane dotyczące jedynie około 100 rozgrywek, a więc jedynie 1\% potrzebnej liczby. Przeprowadzona na ich podstawie analiza nie jest więc zbyt miarodajna, może być potraktowana raczej jako ciekawostka.

Większość użytkowników preferowała wybór zasad niepozwalających statkom się stykać - dlatego tylko ten przypadek będzie analizowany.

W tabeli 7.14 oraz obrazującym ją wykresie z rysunku 7.12 widać, że ogólna tendencja skuteczności algorytmów jest podobna. Widać jej stopniowy wzrost, oraz spadek średniej liczby tur potrzebnych do pokonania gracza. Anomalią jest znaczny wzrost skuteczności w przypadku algorytmu \emph{Max. zysk}, ale wynika to najprawdopodobniej z małej próby badawczej. Dodatkowym czynnikiem utrudniającym miarodajną analizę wyników starć algorytm-gracz jest fakt, że nie każdy gracz gra na takim samym poziomie. Teoretycznie słabszy algorytm może mieć zawyżoną skuteczność, jeśli grało z nim  więcej 'słabszych' graczy. Dlatego konieczna jest większa ilość rozgrywek, aby wyeliminować wpływ tych czynników na ogół danych.

\begin{table}[!h]
    \label{fig:vs-people}
    \centering \includegraphics[width=0.9\linewidth]{img/vs_people.PNG}
    \caption{Skuteczność oraz średnia liczba ruchów algorytmów w starciach z graczami.}
\end{table}

\begin{table}[!h]
    \label{fig:vs-people-chart}
    \centering \includegraphics[width=0.9\linewidth]{img/vs_people_chart.png}
    \caption{Wykres przedstawiający skuteczność oraz średnią liczbę ruchów algorytmów w starciach z graczami.}
\end{table}
\newpage % Rozdziały zaczynamy od nowej strony.
\section{Wnioski}

Na podstawie wyników przedstawionych w poprzednim rozdziale, możemy uszeregować algorytmy od najmniej do najbardziej skutecznego:\begin{enumerate}
    \item Algorytm losowy
    \item Rozszerzony algorytm losowy
    \item Algorytm oparty na heurystyce najbardziej prawdopodobnej lokalizacji na podstawie trafień
    \item Algorytm oparty na heurystyce maksymalizacji zysku priorytetyzującej dłuższe statki
    \item Algorytm oparty na heurystyce maksymalizacji zysku ze strzału
        \item Algorytm oparty na heurystykach maksymalizacji zysku oraz najbardziej prawdopodobnej lokalizacji na podstawie trafień priorytetyzującej dłuższe statki
    \item Algorytm oparty na heurystykach maksymalizacji zysku oraz najbardziej prawdopodobnej lokalizacji na podstawie trafień
\end{enumerate}

Na wykresie widocznym na rysunku 7.1. można zauważyć że heurystyka maksymalizacji zysku ze strzału jest bardziej skuteczna niż heurystyką najbardziej prawdopodobnej lokalizacji na podstawie trafień. Tak więc minimalizacja możliwych położeń statków na mapie jest bardziej opłacalna, nawet jeśli algorytm nie potrafi analizować trafień w celu 'dobijania' statków. Generalnie, z wykresu wyraźnie wynika, że im mniej losowości w algorytmie, tym jest skuteczniejszy - algorytm \emph{Analiza trafień} losowo wybiera ostrzeliwane komórki, jeśli nie ma żadnego trafienia, które może analizować.

Z wykresu wynika także brak znaczącej różnicy pomiędzy algorytmami priorytetyzującymi dłuższe statki, a podstawowymi wariantami tychże algorytmów. Jest to prawdopodobnie spowodowane tym, że trafienie największego statku zazwyczaj nie jest największym wyzwaniem w rozgrywce. Najbardziej problematyczne jest raczej trafienie najmniejszego statku - żadna heurystyka nie jest w stanie pomóc w odnalezieniu jego lokalizacji.

Ze zgromadzonych danych wynika też, że gdy statki mogą się ze sobą stykać, wyniki starć pomiędzy poszczególnymi algorytmami są bardziej wyrównane. Algorytmy losowe nadal są zdecydowanie najsłabsze, ale algorytmy oparte na heurystykach mają skuteczności w przedziale około 40-60\%. W przypadku tego wariantu zasad, na skuteczności wiele tracą algorytmy oparte na heurystyce maksymalizacji zysku ze strzału. Mniej ograniczeń dotyczących lokalizacji statku przez zasady, oznacza że istnieje więcej potencjalnych rozmieszczeń statków na planszy przeciwnika. Algorytm potrzebuje więcej ruchów, aby eliminować, te możliwe położenia. 

Teorię tę potwierdza analiza skuteczności wszystkich algorytmów przeciwko algorytmowi losowemu, która jest widoczna na rysunku 7.2. Łatwo zauważyć, że prawie we wszystkich przypadkach, skuteczności  znacznie różnią się zależnie od wybranych zasad. Jedynym wyjątkiem jest algorytm oparty na heurystyce najbardziej prawdopodobnej lokalizacji na podstawie trafień. Algorytm ten wybiera losowo komórki do ostrzału, jeśli na planszy nie ma żadnego trafienia - nie ma więc na niego wpływu fakt, że statki mogą się ze sobą stykać.

Różnicę dobrze widać też na rysunku 7.2, który przedstawia średnią liczbę tur w zależności od tego z jakim algorytmem mierzył się algorytm losowy. W przypadku wszystkich algorytmów poza \emph{Analizą trafień}, liczba tur gdy statki mogą się ze sobą stykać jest znacznie wyższa niż w przeciwnym wypadku. Widać też że liczba tur spada znacznie wolniej wraz ze wzrostem zaawansowania algorytmu. Wynika to z wcześniej wspomnianego faktu, że heurystyka maksymalizacji zysku ze strzału traci na efektywności gdy statki mogą się ze sobą stykać.

Z tabel 7.1-7.9 wynika, że im skuteczniejszy jest dany algorytm, tym krótsze są średnio jego rozgrywki. Średnio najmniej tur potrzebnych było w starciach algorytmów \emph{Max. zysk + analiza trafień} oraz \emph{Max. zysk rozszerzony + analiza trafień}.

W tabelach 7.11 - 7.13 widoczny jest wpływ liczby statków użytych w rozgrywce na skuteczność algorytmów.

Dodanie najmniejszych statków, składających się z tylko jednej komórki znacznie zmiejsza skuteczność bardziej złożonych algorytmów. Heurystyka \emph{Analiza trafień} jest wtedy bezużyteczna - przydaje się ona jedynie gdy chcemy wnioskować gdzie znajdują się pozostałe komórki trafionego statku. W przypadku jednomasztowców taki scenariusz nigdy nie dojdzie do skutku - trafienie równa się zatopieniu. Widać to wyraźnie w tabeli 7.11 - skuteczność przeciwko \emph{Analiza trafień} znacznie wzrosła, a badany algorytm \emph{Max. zysk + Analiza trafień} ma skuteczność ~50\% przeciwko algorytmom \emph{Max. zysk} oraz \emph{Max. zysk rozszerzony}. Wskazuje na to, że część algorytmu oparta na \emph{Analizie trafień} przestaje w tym scenariuszu przynosić korzyści względem pozostałych algorytmów.

Wpływ dodania statków 'średnich', tj. trzymasztowców przedstawiony jest w tabeli 7.12. Tym razem wyniki są bardziej zbliżone, ale znowu widać wzrost skuteczności badanego algorytmu \emph{Max. zysk + Analiza trafień} przeciwko \emph{Analizie trafień}, gdy statki nie mogą się ze sobą stykać. Tym razem jednak nie jest to spowodowane spadkiem efektywności heurystyki \emph{Analizy trafień}, a raczej wzrostem skuteczności \emph{Max. zysk}. Więcej statków oznacza, że w końcowej fazie rozgrywki, istnieje dużo mniej możliwych pozycji pozostałych statków - są one ograniczone pozycjami tych już zatopionych.

Tabela 7.13 przedstawia zmianę skuteczności \emph{Max. zysk + Analiza trafień} po wprowadzeniu dodatkowego cztero- i pięciomasztowca. Wpływ jest bardzo podobny jak w poprzednium przypadku, dla trzymasztowców. Heurystyka \emph{Max. zysk} staje się dużo bardziej pomocna. Ciekawy jest fakt, że pomimo dodanie 9 dodatkowych komórek do zestrzelenia, liczba tur w żadnym przypadku nie wzrasta o więcej niż 7, a przeciwko losowym algorytmom nawet maleje. To samo zjawisko widoczne jest też w tabeli 7.12. Wynika to z faktu, że dodatkowe statki oznaczają nie tylko dodatkowe cele, ale również dodatkowe ograniczenie pozycji pozostałych statków. Dla algorytmów opartych na heurystyce \emph{Max. zysk} większa liczba statków ułatwia wnioskowanie, a co za tym idzie zwiększa ich skuteczność.

Z tabel 7.14 i 7.15 wynika, że rozstawienie statków na planszy nie ma zbyt dużego wpływu na skuteczność algorytmów. W przypadku algorytmów losowych jest to spowodowane faktem, że strzały obu stron mają takie samo prawdopodobieństwo trafienia, niezależnie od tego jak rozstawione są statki. Algorytm \emph{Analiza trafień} jest podobny w tej kwestii - również oddaje losowe strzały, aż trafi przeciwnika. Algorytmy oparte na heurystyce \emph{Max. zysk} prawdopodobnie wystarczająco szybko eliminują potencjalne położenia statków na planszy i tym samym niwelują korzyści wynikające z danego rozstawienia statków.

\subsection{Ograniczenia pracy}
Nie udało się niestety zebrać wystarczająco dużo danych z rozgrywek człowiek-algorytm, aby reprezentatywnie ocenić skuteczność algorytmów w starciach z człowiekiem.

\input{tex/10-podsumowanie}

%--------------------------------------------
% Literatura
%--------------------------------------------
\cleardoublepage % Zaczynamy od nieparzystej strony
\printbibliography

%--------------------------------------------
% Spisy (opcjonalne)
%--------------------------------------------
\newpage
\pagestyle{plain}

% Wykaz symboli i skrótów.
% Pamiętaj, żeby posortować symbole alfabetycznie
% we własnym zakresie. Ponieważ mało kto używa takiego wykazu,
% uznałem, że robienie automatycznie sortowanej listy
% na poziomie LaTeXa to za duży overkill.
% Makro \acronymlist generuje właściwy tytuł sekcji,
% w zależności od języka.
% Makro \acronym dodaje skrót/symbol do listy,
% zapewniając podstawowe formatowanie.
% //AB
% \vspace{0.8cm}
% \acronymlist
% \acronym{EiTI}{Wydział Elektroniki i Technik Informacyjnych}
% \acronym{PW}{Politechnika Warszawska}
% \acronym{WEIRD}{ang. \emph{Western, Educated, Industrialized, Rich and Democratic}}

\listoffigurestoc     % Spis rysunków.
\vspace{1cm}          % vertical space
\listoftablestoc      % Spis tabel.
\vspace{1cm}          % vertical space
\listofappendicestoc  % Spis załączników

% \renewcommand{\thesubsection}{\Alph{subsection}}


% Załączniki
\newpage
\inputencoding{utf8}
\appendix{Najważniejsze fragmenty kodu aplikacji}

Obie aplikacje są dostępne w repozytoriach na GitHubie:
\begin{itemize}
    \item Frontend - \url{https://github.com/kalina559/battleships-game}
    \item Backend - \url{https://github.com/kalina559/battleships-backend}
\end{itemize}

Są również dostępne na płytce dołączonej do pracy, spakowane w formacie .

\tocless\subsection{Frontend}

\tocless\subsubsection{App.vue}

W pliku App.vue, widocznym na listingu 14, znajduję się główna część aplikacji. Języki HTML oraz CSS wykorzystane są do zdefiniowania strony wizualnej, JavaScript zaś implementuje logikę. W poniższym 
App.vue wykorzystuje pozostałe komponenty zdefiniowane w projekcie, między innymi Header.vue, UserGrid.vue czy OpponentGrid.vue. Dzięki temu ma dostęp do zdarzeń emitowanych przez te komponenty, takich jak na przykład ustawienie statku, czy wybór komórki na planszy przeciwnika.

\begin{addmargin}[0mm]{0mm}
\begin{lstlisting}[
    language=JavaScript,
    numbers=left,
    firstnumber=0,
    caption={Komponent UserGrid.vue},
    aboveskip=0pt,
    breaklines=true
]
<template>
  <div id="app">
    <Header />
    <div class="content">
      <div class="language-switcher">
        <span @click="changeLanguage('en')" class="fi fi-gb"
        :title="$t('englishLanguage')"></span>
        <span @click="changeLanguage('pl')" class="fi fi-pl"
        :title="$t('polishLanguage')"></span>
      </div>
      <Menu v-if="!gameStarted" @startGame="startGame" />
      <div v-if="gameStarted" class="grids">
        <div class="phase">{{ $t(gamePhaseText) }}</div>
        <OpponentGrid :ships="opponentShips" :showShips=false
        :shots="playerShots" @cellSelected="handleUserShot"
          :disabled="currentPlayer !== user"
          :feedbackMessage=$t(opponentGridFeedbackMessage) />
        <UserGrid :ships="playerShips" :shots="opponentShots"
        @shipPlaced="onShipPlaced"
          :feedbackMessage=$t(playerGridFeedbackMessage)
          :shipsCanTouch="shipsCanTouch" />
      </div>
      <Help />
      <div v-if="winner" class="modal">
        <div class="modal-content">
          <p>{{ $t(winnerMessage) }}</p>
          <button @click="resetGame">{{ $t('playAgain') }}</button>
        </div>
      </div>
    </div>
  </div>
</template>

<script>
import { v4 as uuidv4 } from 'uuid';
import Header from './components/Header.vue';
import UserGrid from './components/UserGrid.vue';
import OpponentGrid from './components/OpponentGrid.vue';
import Help from './components/Help.vue';
import Menu from './components/Menu.vue';
import GameApi from './api/GameApi';

const FEEDBACK_OPPONENT_PLACEHOLDER = 'feedbackOpponentPlaceholder';
const FEEDBACK_PLAYER_PLACEHOLDER = 'feedbackPlayerPlaceholder';
const FEEDBACK_PLAYER_SINK = 'feedbackPlayerSink';
const FEEDBACK_PLAYER_HIT = 'feedbackPlayerHit';
const FEEDBACK_PLAYER_MISS = 'feedbackPlayerMiss';
const FEEDBACK_OPPONENT_SINK = 'feedbackOpponentSink';
const FEEDBACK_OPPONENT_HIT = 'feedbackOpponentHit';
const FEEDBACK_OPPONENT_MISS = 'feedbackOpponentMiss';

const WAITING_FOR_OPPONENT_SHIPS = 'waitingForOpponentToDeployShips';
const WAITING_FOR_PLAYER_SHIPS = 'waitingForUserToDeployShips';
const PLAYER_TURN = 'yourTurn';
const OPPONENT_TURN = 'opponentsTurn';

const GamePhase = Object.freeze({
    WaitingForOpponentShips: 1,
    WaitingForPlayerShips: 2,
    PlayerTurn: 3,
    OpponentTurn: 4
});

const Player = Object.freeze({
    User: 1,
    Opponent: 2
});

export default {
  name: 'App',
  components: {
    Header,
    UserGrid,
    OpponentGrid,
    Help,
    Menu
  },
  data() {
    return {
      gameStarted: false,
      gamePhase: GamePhase.WaitingForPlayerShips,
      playerShips: [],
      opponentShips: [],
      playerShipsSet: false,
      opponentShipsSet: false,
      currentPlayer: null,
      opponentShots: [],
      playerShots: [],
      winner: null,
      opponentGridFeedbackMessage: FEEDBACK_OPPONENT_PLACEHOLDER,
      playerGridFeedbackMessage: FEEDBACK_PLAYER_PLACEHOLDER,
      sessionId: null,
      shipsCanTouch: false,
      user: Player.User
    };
  },
  computed: {
    gamePhaseText() {
      switch (this.gamePhase) {
        case GamePhase.WaitingForPlayerShips:
          return WAITING_FOR_PLAYER_SHIPS;
        case GamePhase.WaitingForOpponentShips:
          return WAITING_FOR_OPPONENT_SHIPS;
        case GamePhase.PlayerTurn:
          return PLAYER_TURN;
        case GamePhase.OpponentTurn:
          return OPPONENT_TURN;
        default:
          return '';
      }
    },
    winnerMessage() {
      return this.winner === Player.User ? 'userWon' : 'aiWon';
    }
  },
  methods: {
    changeLanguage(lang) {
      this.$i18n.locale = lang;
    },
    generateOrRetrieveSessionId() {
      let sessionId = this.getCookie('sessionId');
      if (!sessionId) {
        sessionId = uuidv4();
        this.setCookie('sessionId', sessionId, 365);
        // Set cookie to expire in 1 year
      }
      this.sessionId = sessionId;
      GameApi.setSessionId(this.sessionId);
      // Set the session ID in the API client
    },
    getCookie(name) {
      const value = `; ${document.cookie}`;
      const parts = value.split(`; ${name}=`);
      if (parts.length === 2) return parts.pop().split(';').shift();
    },
    setCookie(name, value, days) {
      const expires =
        new Date(Date.now() + days * 864e5).toUTCString();
      document.cookie =
        `${name}=${value}; expires=${expires}; path=/`;
    },
    async startGame(shipsCanTouch) {
      this.gameStarted = true;
      try {
        this.opponentShips = await GameApi.getOpponentShips();
        this.opponentShipsSet = true;
        this.shipsCanTouch = shipsCanTouch
        this.checkPhaseTransition();
      } catch (error) {
        console.error('Failed to get opponent ships:', error);
      }
    },
    async onShipPlaced(ships) {
      this.playerShips = ships;
      this.playerShipsSet = this.playerShips.length === 5;

      if (this.playerShipsSet) {
        try {
          await GameApi.setUserShips(ships);
          this.checkPhaseTransition();
        } catch (error) {
          console.error('Failed to set user ships:', error);
        }
      } else {
        this.checkPhaseTransition();
      }
    },
    checkPhaseTransition() {
      if (this.playerShipsSet && this.opponentShipsSet) {
        this.determineStartingPlayer();
      } else if (this.playerShipsSet) {
        this.gamePhase = GamePhase.WaitingForOpponentShips;
      } else if (this.opponentShipsSet) {
        this.gamePhase = GamePhase.WaitingForPlayerShips;
      }
    },
    determineStartingPlayer() {
      const randomStart = Math.random() < 0.5;
      this.currentPlayer = randomStart 
        ? Player.User 
        : Player.Opponent;
      this.gamePhase = randomStart 
        ? GamePhase.PlayerTurn
        : GamePhase.OpponentTurn;
      if (!randomStart) {
        this.opponentMove();
      }
    },
    async opponentMove() {
      if (this.currentPlayer !== Player.Opponent) return;
      try {
        const move = await GameApi.opponentShot();
        await this.updateGameState();
        this.playerGridFeedbackMessage = move.isHit 
            ? (move.isSunk ? FEEDBACK_OPPONENT_SINK
            : FEEDBACK_OPPONENT_HIT) : FEEDBACK_OPPONENT_MISS;

        this.checkIfWinner(move, Player.Opponent);
      } catch (error) {
        console.error('Failed to get opponent move:', error);
      }
    },
    async handleUserShot(x, y) {
      if (this.currentPlayer !== Player.User) return;
      try {
        const move = await GameApi.userShot({ x, y });
        await this.updateGameState();
        this.opponentGridFeedbackMessage = move.isHit
            ? (move.isSunk ? FEEDBACK_PLAYER_SINK
            : FEEDBACK_PLAYER_HIT) : FEEDBACK_PLAYER_MISS;

        this.checkIfWinner(move, Player.User);
      } catch (error) {
        console.error('Failed to handle user shot:', error);
      }
    },
    async checkIfWinner(move, side) {
      if(move.win == true){
          this.winner = side;
        } else {
          this.switchTurn();
        }
    },
    async updateGameState() {
      try {
        const gameState = await GameApi.getGameState();

        this.playerShips
            .splice(0, this.playerShips.length,
                ...gameState.userShips);
        this.opponentShips
            .splice(0, this.opponentShips.length,
                ...gameState.opponentShips);
        this.playerShots
            .splice(0, this.playerShots.length,
                ...gameState.playerShots);
        this.opponentShots
            .splice(0, this.opponentShots.length,
                ...gameState.opponentShots);
        this.shipsCanTouch = gameState.shipsCanTouch;

      } catch (error) {
        console.error('Failed to update game state:', error);
      }
    },
    switchTurn() {
      this.currentPlayer = this.currentPlayer === Player.User
        ? Player.Opponent
        : Player.User;
      this.gamePhase = this.currentPlayer === Player.User
        ? GamePhase.PlayerTurn
        : GamePhase.OpponentTurn;
      if (this.currentPlayer === Player.Opponent) {
        setTimeout(() => {
          this.opponentMove();
        }, 1000);
      }
    },
    resetGame() {
      this.gameStarted = false;
      this.gamePhase = GamePhase.WaitingForPlayerShips;
      this.playerShips = [];
      this.opponentShips = [];
      this.playerShipsSet = false;
      this.opponentShipsSet = false;
      this.currentPlayer = null;
      this.opponentShots = [];
      this.playerShots = [];
      this.winner = null;
      this.opponentGridFeedbackMessage = '';
      this.playerGridFeedbackMessage = '';
    }
  },
  mounted() {
    this.generateOrRetrieveSessionId();
  }
};
</script>

<style>
.content {
  display: flex;
  flex-direction: column;
  align-items: center;
}

.language-switcher {
  display: flex;
  justify-content: center;
  margin-bottom: 10px;
}

.language-switcher img {
  cursor: pointer;
  margin: 0 10px;
}

.grids {
  display: flex;
  flex-direction: column;
  align-items: center;
}

.phase {
  margin-bottom: 20px;
  font-size: 18px;
  font-weight: bold;
}

.modal {
  position: fixed;
  top: 50%;
  left: 50%;
  transform: translate(-50%, -50%);
  background-color: white;
  padding: 20px;
  border: 1px solid #333;
  box-shadow: 0 0 10px rgba(0, 0, 0, 0.5);
  z-index: 1000;
}

.modal-content {
  text-align: center;
}

.language-switcher .fi {
  margin-right: 8px; /* Add spacing between the span elements */
  width: 32px;
  height: 32px;
  cursor: pointer;
}

.language-switcher .fi:last-child {
  margin-right: 0; /* Remove right margin for the last element */
}
</style>

\end{lstlisting}
\end{addmargin}

\tocless\subsubsection{UserGrid.vue}
Komponent UserGrid.vue jest drugim najbardziej złożonym elementem aplikacji frontend. Wynika to z faktu, że konieczne było zaimplementowanie w nim logiki związanej z rozstawianiem statków na planszy gracza. Znajdują się tutaj metody takie jak isOccupied, isAdjacentOccupied, które sprawdzają czy dana pozycja statku jest możliwa do zrealizowania. Użytkownik dostaje informację zwrotną poprzez zmianę koloru statku na kolor zielony lub czerwony. Kolor zielony oznacza, że statek może być postawiony na danej pozycji.

Kontrolowanie zasad byłoby również możliwe poprzez backend, ale uznano że jest to bardzo nieoptymalne rozwiązanie - w trakcie rozstawiania statków gracz może sprawdzać dziesiątki różnych pozycji zanim podejmie swoją decyzję. Odpytywanie tyle razy backendu niepotrzebnie by obciążało aplikację.


\begin{addmargin}[0mm]{0mm}
\begin{lstlisting}[
    language=JavaScript,
    numbers=left,
    firstnumber=0,
    caption={Główny komponent aplikacji App.vue},
    aboveskip=0pt,
    breaklines=true
]
<template>
  <div class="grid">
    <h2>{{ $t('playersGrid') }}</h2>
    <div class="feedback" 
        v-if="feedbackMessage">{{ feedbackMessage }}</div>
    <div class="grid-container" @keydown="handleKeydown"
        tabindex="0" ref="gridContainer">
      <div class="row">
        <div class="corner"></div>
        <div v-for="label in columnLabels" :key="label"
            class="column-label">{{ label }}</div>
      </div>
      <div v-for="(row, rowIndex) in rows" :key="rowIndex" 
        class="row">
        <div class="row-label">{{ rowLabels[rowIndex] }}</div>
        <div v-for="cell in row" :key="cell.id" 
            :class="['cell', getCellClass(rowIndex, cell.id)]">
          <span v-if="isMissCell(rowIndex, cell.id)"
            class="miss-marker">X</span>
        </div>
      </div>
    </div>
    <div class="controls">
      <div class="control-row">
        <button @click="handleKeydown({ key: 'ArrowUp' })">
            ↑
        </button>
      </div>
      <div class="control-row">
        <button @click="handleKeydown({ key: 'ArrowLeft' })">
            ←
        </button>
        <button @click="handleKeydown({ key: 'ArrowDown' })">
            ↓
        </button>
        <button @click="handleKeydown({ key: 'ArrowRight' })">
            →
        </button>
      </div>
      <div class="control-row">
        <button @click="handleKeydown({ key: 'r' })">
        {{ $t('rotateButton') }}
        </button>
            <button @click="handleKeydown({ key: 'Enter' })">
        {{ $t('deployButton') }}
        </button>
      </div>
    </div>
  </div>
</template>


<script>
export default {
  name: 'UserGrid',
  props: {
    ships: {
      type: Array,
      default: () => []
    },
    shots: {
      type: Array,
      default: () => []
    },
    feedbackMessage: {
      type: String,
      default: ''
    },
    shipsCanTouch: {
      type: Boolean,
      default: false
    },
  },
  data() {
    return {
      // eslint-disable-next-line no-unused-vars
      rows: Array.from({ length: 10 }, (_, rowIndex) =>
        Array.from({ length: 10 }, (__, cellIndex) => ({
          id: cellIndex,
          label: ''
        }))
      ),
      rowLabels: 'ABCDEFGHIJ'.split(''),
      columnLabels: Array.from({ length: 10 }, (_, i) => i + 1),
      placedShips: [],
      currentShip: { size: 5, coordinates: [] },
      currentShipDirection: 'horizontal',
      currentShipPosition: { x: 5, y: 5 }
    };
  },
  mounted() {
    this.initCurrentShip();
    window.addEventListener('keydown', this.handleKeydown);
  },
  beforeUnmount() {
    window.removeEventListener('keydown', this.handleKeydown);
  },
  methods: {
    initCurrentShip() {
      this.updateCurrentShipCoordinates();
    },
    handleKeydown(event) {
      if (this.currentShip.size === 0) return;
      const key = event.key;
      switch (key) {
        case 'ArrowUp':
          this.moveShip(-1, 0);
          break;
        case 'ArrowDown':
          this.moveShip(1, 0);
          break;
        case 'ArrowLeft':
          this.moveShip(0, -1);
          break;
        case 'ArrowRight':
          this.moveShip(0, 1);
          break;
        case 'r':
        case 'R':
          this.rotateShip();
          break;
        case 'Enter':
          this.placeShip();
          break;
      }
    },
    moveShip(dx, dy) {
      const newPosition = { x: this.currentShipPosition.x + dx,
        y: this.currentShipPosition.y + dy };
      this.currentShipPosition = newPosition;
      this.updateCurrentShipCoordinates();
    },
    rotateShip() {
      const newDirection =
        this.currentShipDirection === 'horizontal'
            ? 'vertical'
            : 'horizontal';
      this.currentShipDirection = newDirection;
      this.updateCurrentShipCoordinates();
    },
    isValidPosition(position, size, direction) {
      for (let i = 0; i < size; i++) {
        const x = direction === 'horizontal'
            ? position.x
            : position.x + i;
        const y = direction === 'horizontal'
            ? position.y + i 
            : position.y;

        if (x < 0 || x >= 10 || y < 0 || y >= 10 
            || this.isOccupied(x, y)
            || ( !this.shipsCanTouch
                && this.isAdjacentOccupied(x, y))) {
          return false;
        }
      }
      return true;
    },
    isOccupied(x, y) {
      return this.placedShips.some(ship =>
        ship.coordinates.some(coord => coord.x === x && coord.y === y)
      );
    },
    isAdjacentOccupied(x, y) {
      const adjacentOffsets = [
        { dx: -1, dy: -1 }, { dx: -1, dy: 0 }, { dx: -1, dy: 1 },
        { dx: 0, dy: -1 }, { dx: 0, dy: 1 },
        { dx: 1, dy: -1 }, { dx: 1, dy: 0 }, { dx: 1, dy: 1 }
      ];

      return adjacentOffsets.some(offset => {
        const adjacentX = x + offset.dx;
        const adjacentY = y + offset.dy;
        return (
          adjacentX >= 0 && adjacentX < 10 &&
          adjacentY >= 0 && adjacentY < 10 &&
          this.isOccupied(adjacentX, adjacentY)
        );
      });
    },
    updateCurrentShipCoordinates() {
      this.currentShip.coordinates =
        Array.from({ length: this.currentShip.size }, (_, i) => {
        return this.currentShipDirection === 'horizontal'
          ? { x: this.currentShipPosition.x,
            y: this.currentShipPosition.y + i }
          : { x: this.currentShipPosition.x + i,
            y: this.currentShipPosition.y };
      });
    },
    placeShip() {
      if (this.isValidPosition(this.currentShipPosition,
        this.currentShip.size,
        this.currentShipDirection)) {
        this.placedShips.push({ ...this.currentShip });
        this.$emit('shipPlaced', this.placedShips,
            this.currentShip.size);
        this.updateNextShip();
      }
    },
    updateNextShip() {
      if (this.currentShip.size > 1) {
        this.currentShip.size--;
      } else {
        this.currentShip = { size: 0, coordinates: [] };
        // No more ships to place
      }
      this.currentShipPosition = { x: 5, y: 5 };
      // Reset to middle
      this.updateCurrentShipCoordinates();
    },
    getCellClass(rowIndex, cellIndex) {
      if (this.isShotCell(rowIndex, cellIndex)) {
        return this.isHitCell(rowIndex, cellIndex)
            ? 'hit-cell'
            : 'miss-cell';
      }
      if (this.isShipCell(rowIndex, cellIndex)) {
        return 'ship-cell';
      }
      if (this.isCurrentShipCell(rowIndex, cellIndex)) {
        return this.isValidPosition(this.currentShipPosition,
            this.currentShip.size,
            this.currentShipDirection)
                ? 'valid-ship'
                : 'invalid-ship';
      }
      return '';
    },
    isShotCell(rowIndex, cellIndex) {
      return this.shots.some(shot => shot.x === rowIndex
        && shot.y === cellIndex);
    },
    isHitCell(rowIndex, cellIndex) {
      return this.shots.some(shot => shot.x === rowIndex
        && shot.y === cellIndex && shot.isHit);
    },
    isMissCell(rowIndex, cellIndex) {
      const shot = this.shots.find(shot => shot.x === rowIndex
        && shot.y === cellIndex);
      return shot && !shot.isHit;
    },
    isShipCell(rowIndex, cellIndex) {
      return this.placedShips.some(ship =>
        ship.coordinates.some(coord => coord.x === rowIndex
            && coord.y === cellIndex)
      );
    },
    isCurrentShipCell(rowIndex, cellIndex) {
      return this.currentShip.coordinates
        .some(coord => coord.x === rowIndex
        && coord.y === cellIndex);
    }
  }
};
</script>


<style scoped>
.grid {
  display: flex;
  flex-direction: column;
  align-items: center;
}

.feedback {
  font-size: 18px;
  font-weight: bold;
  margin-bottom: 10px;
}

.grid-container {
  display: grid;
  grid-template-columns: 30px repeat(10, 30px);
  grid-template-rows: 30px repeat(10, 30px);
  gap: 2px;
}

.grid-container:focus {
  outline: none;
}

.row {
  display: contents;
}

.cell,
.column-label,
.row-label {
  width: 30px;
  height: 30px;
  display: flex;
  align-items: center;
  justify-content: center;
}

.cell {
  background-color: lightblue;
  border: 1px solid #333;
}

.valid-ship {
  background-color: lightgreen;
}

.invalid-ship {
  background-color: lightcoral;
}

.hit-cell {
  background-color: red;
}

.miss-cell {
  background-color: white;
  position: relative;
}

.miss-marker {
  color: black;
  font-size: 24px;
  position: absolute;
}

.ship-cell {
  background-color: darkblue;
}

.column-label,
.row-label {
  background-color: #f0f0f0;
  font-weight: bold;
}

.corner {
  background-color: #f0f0f0;
}

.controls {
  display: flex;
  flex-direction: column;
  align-items: center;
  margin-top: 20px;
}

.control-row {
  display: flex;
  justify-content: center;
  margin: 5px 0;
}

.controls button {
  margin: 0 5px;
  padding: 10px 20px;
  font-size: 16px;
}
</style>
\end{lstlisting}
\end{addmargin}


\tocless\subsubsection{GameApi}
Plik GameApi.js zawiera definicje metod wykorzystywanych do odpytywania aplikacji backend poprzez zapytania AJAX. W tym celu wykorzystano klienta Axios. Do każdego zapytania dołączany jest header "X-Session-Id", który potrzebny jest do identyfikacji danej sesji gry.

\begin{addmargin}[0mm]{0mm}
\begin{lstlisting}[
    language={JavaScript},
    numbers=left,
    firstnumber=5,
    caption={Plik GameApi.js},
    aboveskip=0pt
]
import axios from 'axios';

const apiClient = axios.create({
  baseURL: process.env.VUE_APP_API_URL,
  withCredentials: true,
  // Include credentials (cookies) with requests
  headers: {
    'Content-Type': 'application/json'
  }
});

let sessionId = null;

function setSessionId(id) {
  sessionId = id;
}

apiClient.interceptors.request.use(config => {
  if (sessionId) {
    config.headers['X-Session-Id'] = sessionId;
  }
  return config;
});

export default {
  setSessionId,
  async getAiTypes(shipsCanTouch) {
    const response = await apiClient.post('/AiType/list',
        shipsCanTouch);
    return response.data;
  },
  async selectAiType(type) {
    await apiClient.post('/AiType/select', type);
  },
  async updateRules(shipsCanTouch) {
    await apiClient.post('/Rules/update', shipsCanTouch);
  },
  async getOpponentShips() {
    const response = await apiClient.get('/ShipLocations/opponent');
    return response.data;
  },
  async setUserShips(shipData) {
    await apiClient.post('/ShipLocations/user', shipData);
  },
  async userShot(position) {
    const response = await apiClient.post('shot/user', position);
    return response.data;
  },
  async opponentShot() {
    const response = await apiClient.get('shot/opponent');
    return response.data;
  },
  async getGameState() {
    const response = await apiClient.get('gamestate/get');
    return response.data;
  },
  async clearGameState() {
    await apiClient.get('gamestate/clear');
  }
};

\end{lstlisting}
\end{addmargin}

\tocless\subsection{Backend}

\tocless\subsubsection{Program.cs}

Najważniejszy element aplikacji backend. Służy do konfiguracji aplikacji, na przykład poprzez pozwolenie na CORS (Cross-Origin Resource Sharing) - dzięki temu frontend może wykonywać zapytania do backendu pomimo tego, że oba znajdują się w innych domenach. Program.cs umożliwia również rejestrowania serwisów, co jest konieczne aby później korzystać ze wstrzykiwania zależności (Dependency injection). 

\begin{addmargin}[0mm]{0mm}
\begin{lstlisting}[
    language={[Sharp]C},
    numbers=left,
    firstnumber=0,
    caption={Klasa ShotController.cs},
    aboveskip=0pt
]
using Battleships.Common.Settings;
using Battleships.Core.Services;
using Battleships.Services.Interfaces;
using Battleships.WebApi;

var builder = WebApplication.CreateBuilder(args);

// Add services to the container.
builder.Services.AddControllers();
builder.Services.AddMemoryCache();
builder.Services.AddHttpContextAccessor();
builder.Services.AddDistributedMemoryCache();
builder.Services.AddSession(options =>
{
    options.IdleTimeout = TimeSpan.FromMinutes(30);
    options.Cookie.HttpOnly = true;
    options.Cookie.IsEssential = true;
    options.Cookie.SameSite = SameSiteMode.None;
    // to allow cross-site requests
    options.Cookie.SecurePolicy = CookieSecurePolicy.Always;

});

builder.Services.AddEndpointsApiExplorer();
builder.Services.AddSwaggerGen();

builder.Services.AddCors(options =>
{
    options.AddPolicy("AllowSpecificOrigin",
        builder =>
        {
            builder.WithOrigins("http://localhost:8080",
                "https://kalina559.github.io") // frontend URL
                   .AllowAnyMethod()
                   .AllowAnyHeader()
                   .AllowCredentials();

        });
});

builder.Services.AddScoped<IShipLocationService,
    ShipLocationService>();
builder.Services.AddScoped<IGenerateMoveService,
    GenerateMoveService>();
builder.Services.AddScoped<IGameStateService,
    GameStateService>();
builder.Services.AddScoped<IAiTypeService,
    AiTypeService>();
builder.Services.AddScoped<IRuleTypeService,
    RuleTypeService>();
builder.Services.AddScoped<ICosmosDbService,
    CosmosDbService>();

var cosmosDbSettings = new CosmosDbSettings();
builder.Configuration.Bind(nameof(CosmosDbSettings),
    cosmosDbSettings);
builder.Services.AddSingleton(cosmosDbSettings);

ServiceCollectionSetup.InitializeCosmosClientInstanceAsync(
cosmosDbSettings, builder.Services);


builder.Logging.ClearProviders();
builder.Logging.AddConsole();

var app = builder.Build();

app.UseSwagger();
app.UseSwaggerUI();
app.UseDeveloperExceptionPage();

if (app.Environment.IsDevelopment())
{
    app.UseDeveloperExceptionPage();
}

app.UseHttpsRedirection();
app.UseRouting();
app.UseCors("AllowSpecificOrigin");
app.UseSession();
app.UseAuthorization();

app.MapControllers();

app.Run();
\end{lstlisting}
\end{addmargin}

\tocless\subsubsection{Przykładowy endpoint RESTful API}

\begin{addmargin}[0mm]{0mm}
\begin{lstlisting}[
    language={[Sharp]C},
    numbers=left,
    firstnumber=0,
    caption={Klasa ShotController},
    aboveskip=0pt
]

using Battleships.Common.GameClasses;
using Battleships.Services.Interfaces;
using Microsoft.AspNetCore.Mvc;

namespace Battleships.WebApi.Controllers
{
    [ApiController]
    [Route("api/[controller]")]
    public class ShotController(
        IGameStateService gameStateService,
        IGenerateMoveService opponentMoveService)
        : ControllerBase
    {
        [HttpPost("user")]
        public IActionResult UserShot([FromBody] Shot shot)
        {
            var result = gameStateService
                .ProcessShot(shot.X, shot.Y, isPlayer: true);
            var win = gameStateService.CheckWinCondition();
            return Ok(new { result.IsHit,
                result.IsSunk,
                Win = win });
        }

        [HttpGet("opponent")]
        public IActionResult OpponentShot()
        {
            var gameState = gameStateService.GetGameState();
            var (X, Y) = opponentMoveService.GenerateMove(
                gameState.OpponentShots,
                gameState.UserShips,
                gameState.ShipsCanTouch,
                gameState.OpponentAiType);

            var result = gameStateService
                .ProcessShot(X, Y, isPlayer: false);
            var win = gameStateService.CheckWinCondition();
            return Ok(
                new { 
                    result.IsHit,
                    result.IsSunk,
                    Win = win });
        }
    }
}

\end{lstlisting}
\end{addmargin}


\tocless\subsubsection{GameStateService}

GameStateService to klasa, która obsługuje kluczowe elementy rozgrywki. Za pomocą MemoryCache zapisuje i odczytuje instancje GameState, które trzymają informację o rozgrywce dla danego SessionId. Tutaj odbywa się też procesowanie strzałów, a więc sprawdzanie czy któryś ze statków gracza lub przeciwnika został trafiony. GameStateService kontroluje również czy rozgrywka nie została zakończona, ponieważ któraś ze stron zatopiła już wszystkie statki przeciwnika - jeśli tak, to przebieg gry zostaje zapisany jako dokument JSON w bazie danych Cosmos DB.

\begin{addmargin}[0mm]{0mm}
\begin{lstlisting}[
    language={[Sharp]C},
    numbers=left,
    firstnumber=0,
    caption={Klasa GameStateService},
    aboveskip=0pt
]
using Battleships.Common.CosmosDb;
using Battleships.Common.GameClasses;
using Battleships.Core.Exceptions;
using Battleships.Services.Interfaces;
using Microsoft.AspNetCore.Http;
using Microsoft.Extensions.Caching.Memory;
using Microsoft.Extensions.Logging;
using Newtonsoft.Json;
using Microsoft.Extensions.Hosting;
using Microsoft.AspNetCore.Hosting;

namespace Battleships.Core.Services
{
    public class GameStateService(
        IHttpContextAccessor httpContextAccessor,
        ILogger<GameStateService> logger,
        IMemoryCache memoryCache,
        ICosmosDbService cosmosDbService,
        IHostEnvironment env) : IGameStateService
    {
        public GameState GetGameState()
        {
            var sessionId = httpContextAccessor.HttpContext
                .Request.Headers["X-Session-Id"].ToString();

            if (!memoryCache.TryGetValue(sessionId, out string? gameStateJson))
            {
                logger.LogInformation(
                    "No game state found in cache for session ID: {sessionId}",
                        sessionId);
                return new GameState();
            }           

            var deserializedGameState = DeserializeGameStateJson(gameStateJson);

            logger.LogInformation(
                "Game state retrieved from cache for session ID: {sessionId}",
                    sessionId);
            return deserializedGameState;
        }

        public void SaveGameState(GameState gameState)
        {
            var sessionId = httpContextAccessor.HttpContext
            .Request.Headers["X-Session-Id"].ToString();
            if (string.IsNullOrEmpty(sessionId))
            {
                logger.LogWarning(
                    "Session ID is missing in the request headers.");
                return;
            }

            var gameStateJson = JsonConvert.SerializeObject(gameState);
            memoryCache.Set(sessionId, gameStateJson);
            logger.LogInformation(
                "Game state saved in cache for session ID: {sessionId}",
                    sessionId);
        }

        public ShotResult ProcessShot(int x, int y, bool isPlayer)
        {
            var gameState = GetGameState();
            var targetShips = isPlayer
                ? gameState.OpponentShips
                : gameState.UserShips;
            var shotList = isPlayer 
                ? gameState.PlayerShots 
                : gameState.OpponentShots;

            var hit = targetShips.Any(
                ship => ship.Coordinates
                    .Any(coord => coord.X == x
                        && coord.Y == y));
            shotList.Add(new Shot { X = x, Y = y,
                IsHit = hit });
            bool isSunk = false;

            if (hit)
            {
                var ship = targetShips.First(
                    s => s.Coordinates
                        .Any(coord => coord.X == x
                            && coord.Y == y));
                var shipSank =
                    ship.Coordinates.All(
                    coord => shotList
                        .Any(shot => shot.X == coord.X
                            && shot.Y == coord.Y));
                if (shipSank)
                {
                    ship.IsSunk = true;
                    isSunk = true;
                }
            }

            SaveGameState(gameState);

            return new ShotResult { IsHit = hit,
                IsSunk = isSunk };
        }

        public bool CheckWinCondition(bool testMode = false)
        {
            var gameState = GetGameState();

            var allPlayerShipsSunk =
                gameState.UserShips.All(ship => ship.IsSunk);
            var allOpponentShipsSunk =
                gameState.OpponentShips.All(ship => ship.IsSunk);

            if (allPlayerShipsSunk || allOpponentShipsSunk)
            {
                SaveGameSessionToDb(gameState,
                    playerWon: allOpponentShipsSunk, testMode);
                return true;
            }

            return false;
        }

        private void SaveGameSessionToDb(GameState gameState,
            bool playerWon, bool testMode)
        {
            if (env.IsProduction() || testMode)
            {
                var gameSession = new GameSession
                {
                    Id = Guid.NewGuid(),
                    GameStateJson = JsonConvert.SerializeObject(gameState),
                    SessionId =
                        httpContextAccessor.HttpContext.
                            Request.Headers["X-Session-Id"].ToString(),
                    DateCreated = DateTime.UtcNow,
                    PlayerAiType = (int?)gameState.PlayerAiType,
                    OpponentAiType = (int)gameState.OpponentAiType,
                    ShipsCanTouch = gameState.ShipsCanTouch,
                    OpponentShipsSunk =
                        gameState.OpponentShips
                            .Where(x => x.IsSunk).Count(),
                    PlayersShipsSunk =
                        gameState.UserShips
                            Where(x => x.IsSunk).Count(),
                    OpponentMovesCount = gameState.OpponentShots.Count(),
                    PlayerMovesCount = gameState.PlayerShots.Count(),
                    PlayerWon = playerWon
                };

                cosmosDbService.AddGameSessionAsync(gameSession, testMode);
            }
        }

        public void ClearGameState()
        {
            var gameState = new GameState
            {
                UserShips = [],
                OpponentShips = [],
                PlayerShots = [],
                OpponentShots = []
            };
            SaveGameState(gameState);
        }

        private static GameState DeserializeGameStateJson(string? gameStateJson)
        {
            if (gameStateJson == null)
            {
                throw new NullGameStateException(
                    $"Game state JSON was found but was null.");
            }

            var gameState = JsonConvert
                .DeserializeObject<GameState>(gameStateJson);

            return gameState ??
                throw new NullGameStateException(
                    $"Game state JSON was found but was null after deserialization");
        }
    }
}
\end{lstlisting}
\end{addmargin}


\tocless\subsubsection{GridHelper}

GridHelper jest statyczną klasą pomocniczą, która zapewnia wiele metod używanych przez algorytmy do analizowania planszy przeciwnika. Umożliwia między innymi sprawdzanie czy dana komórka jest dostępna do ostrzału, czy jest częścią zatopionego statku, czy jest trafiona, etc.

\begin{addmargin}[0mm]{0mm}
\begin{lstlisting}[
    language={[Sharp]C},
    numbers=left,
    firstnumber=0,
    caption={Klasa GridHelper},
    aboveskip=0pt
]
using Battleships.Common.GameClasses;
using System.Diagnostics;

namespace Battleships.Common.Helpers
{
    public static class GridHelper
    {
        public static IEnumerable<(int X, int Y)> GetAllAdjacentCells(
            int x, int y)
        {
            return GetSideAdjacentCells(x, y)
                .Concat(GetEdgeAdjacentCells(x, y));
        }

        /// <summary>
        /// Gets all cells that are touching the sides
        /// of the cell with (x,y) coordinates
        /// </summary>
        /// <param name="x"></param>
        /// <param name="y"></param>
        /// <returns></returns>
        public static IEnumerable<(int X, int Y)> 
            GetSideAdjacentCells(int x, int y)
        {
            return new List<(int X, int Y)>
    {
        (x - 1, y), (x + 1, y),
        (x, y - 1), (x, y + 1)
    };
        }

        /// <summary>
        /// Gets all cells that are touching
            the edges of the cell with (x,y) coordinates
        /// </summary>
        /// <param name="x"></param>
        /// <param name="y"></param>
        /// <returns></returns>
        public static IEnumerable<(int X, int Y)> GetEdgeAdjacentCells(
            int x, int y)
        {
            return new List<(int X, int Y)>
    {
        (x - 1, y - 1), (x - 1, y + 1),
        (x + 1, y - 1), (x + 1, y + 1)
    };
        }

        public static bool IsValidShipPosition(
            List<Shot> previousShots,
            List<Ship> opponentShips,
            bool shipsCanTouch,
            int x, int y,
            int length,
            bool isVertical)
        {
            int maxLength = isVertical ? x + length : y + length;
            if (maxLength > 10) return false;

            for (int i = 0; i < length; i++)
            {
                int currentX = isVertical ? x + i : x;
                int currentY = isVertical ? y : y + i;

                if (!shipsCanTouch && IsEdgeAdjacentCellHit(
                    previousShots,
                    currentX,
                    currentY))
                {
                    return false;
                }

                if (!(IsCellAvailable(previousShots, currentX, currentY)
                      || (IsCellHit(previousShots, currentX, currentY)
                        && !IsPartOfSunkShip(currentX, currentY, opponentShips))))
                {
                    return false;
                }
            }

            return true;
        }

        public static bool IsCellAvailable(List<Shot> previousShots, int x, int y)
        {
            var shot = previousShots.FirstOrDefault(s => s.X == x && s.Y == y);
            return shot == null;
        }

        public static bool IsWithinBounds(int x, int y)
        {
            return x >= 0 && x < 10 && y >= 0 && y < 10;
        }

        public static bool 
            IsPartOfSunkShip(int x, int y, List<Ship> opponentShips)
        {
            return opponentShips
                .Any(ship => ship.IsSunk && ship.Coordinates
                    .Any(coord => coord.X == x && coord.Y == y));
        }        

        public static bool IsCellHit(List<Shot> previousShots, int x, int y)
        {
            var shot = previousShots.FirstOrDefault(s => s.X == x && s.Y == y);
            return shot != null && shot.IsHit;
        }

        public static bool IsEdgeAdjacentCellHit(
            List<Shot> previousShots, int x, int y)
        {
            var edgeAdjacentCells = GetEdgeAdjacentCells(x, y);
            var hit = edgeAdjacentCells
                .Any(cell => previousShots
                    .Any(shot => cell.X == shot.X 
                        && cell.Y == shot.Y && shot.IsHit));

            return hit;
        }

        public static void PrintProbabilityGrid(
            int[] probabilityMap,
            int rows, int columns)
        {
            int[,] grid = new int[rows, columns];

            var mapElements =
                probabilityMap.Select((prob, index) => (prob, index)).ToList();

            foreach (var (probability, index) in mapElements)
            {

                int y = index / 10;
                int x = index % 10;
                grid[y, x] = probability;
            }

            for (int y = 0; y < rows; y++)
            {
                for (int x = 0; x < columns; x++)
                {
                    Console.Write($"{grid[x, y],4}");
                }
                Console.WriteLine();
            }
        }

        public static void PrintGrid(bool[,] grid, int rows, int columns)
        {

            for (int y = 0; y < rows; y++)
            {
                for (int x = 0; x < columns; x++)
                {
                    Debug.Write($"{Convert.ToInt32(grid[x, y]),4}");
                }

                Debug.Write("\n");
            }

            Debug.Write("\n");
        }

        public static void OrderHitCluster(ref List<(int X, int Y)> cluster)
        {
            cluster = [.. cluster.OrderBy(cell => cell.Y)
                .ThenBy(cell => cell.X)];
        }
    }
}

\end{lstlisting}
\end{addmargin}





\tocless\subsubsection{HeuristicHelper}

HeuristicHelper jest statyczna klasą pomocniczą, podobnie jak GridHelper z rodziału 1.2.4. Metody zawarte w tej klasie są jednak bardziej bezpośrednio związane z algorytmami decyzyjnymi. Metody te pozwalają na zwiększanie wartości komórek na mapie prawdopodobieństwa w zależności od tego ile wariacji statków może znajdować się na danej komórce, czy sąsiaduje z trafieniem bądź serią trafień, etc.

\begin{addmargin}[0mm]{0mm}
\begin{lstlisting}[
    language={[Sharp]C},
    numbers=left,
    firstnumber=0,
    caption={Klasa HeuristicHelper},
    aboveskip=0pt
]

using Battleships.Common.GameClasses;

namespace Battleships.Common.Helpers
{
    public static class HeuristicHelper
    {
        public static void AdjustProbabilityForShipLocations(
            List<Shot> previousShots,
            List<Ship> opponentShips,
            bool shipsCanTouch,
            int[] probabilityMap,
            int weight,
            bool dynamicWeight = false,
            int dynamicPower = 1)
        {
            var remainingShipLengths = opponentShips
                .Where(ship => !ship.IsSunk)
                .Select(ship => ship.Coordinates.Count)
                .ToList();

            foreach (var length in remainingShipLengths)
            {
                for (int x = 0; x < 10; x++)
                {
                    for (int y = 0; y < 10; y++)
                    {
                        if (GridHelper.IsValidShipPosition(
                            previousShots,
                            opponentShips,
                            shipsCanTouch, x, y,
                            length, isVertical: true))
                        {
                            for (int i = 0; i < length; i++)
                            {
                                probabilityMap[(y * 10) + x + i]
                                += dynamicWeight 
                                    ? weight * (int)Math.Pow(length, dynamicPower) 
                                    : weight;
                            }
                        }

                        if (GridHelper.IsValidShipPosition(
                            previousShots,
                            opponentShips,
                            shipsCanTouch, x, y,
                            length, isVertical: false))
                        {
                            for (int i = 0; i < length; i++)
                            {
                                probabilityMap[((y + i) * 10) + x]
                                += dynamicWeight
                                ? weight * (int)Math.Pow(length, dynamicPower)
                                : weight;
                            }
                        }
                    }
                }
            }
        }

        public static int AdjustProbabilityForHitClusters(
            List<Shot> previousShots,
            List<Ship> opponentShips,
            int[] probabilityMap,
            int singleHitWeight,
            int clusterWeight)
        {
            var hitClusters = FindHitClusters(previousShots, opponentShips);
            foreach (var cluster in hitClusters)
            {
                if (cluster.Count > 1)
                {
                    var (startX, startY) = cluster.First();
                    var (endX, endY) = cluster.Last();

                    var isHorizontal = startX == endX;

                    if (isHorizontal)
                    {
                        // Increase probability for cells
                        // extending the horizontal cluster
                        IncreaseProbabilityForCell(
                            previousShots,
                            startX, startY - 1,
                            probabilityMap, 
                            clusterWeight);
                        IncreaseProbabilityForCell(
                            previousShots,
                            endX, endY + 1,
                            probabilityMap,
                            clusterWeight);
                    }
                    else
                    {
                        // Increase probability for cells
                        // extending the vertical cluster
                        IncreaseProbabilityForCell(previousShots,
                            startX - 1,
                            startY,
                            probabilityMap, 
                            clusterWeight);
                        IncreaseProbabilityForCell(
                            previousShots,
                            endX + 1, endY,
                            probabilityMap,
                            clusterWeight);
                    }
                }
                else
                {
                    AdjustProbabilityForSingleHit(
                        previousShots,
                        probabilityMap,
                        cluster.First(),
                        singleHitWeight);
                }
            }

            return hitClusters.Count();
        }

        public static List<List<(int X, int Y)>> FindHitClusters(
            List<Shot> previousShots, List<Ship> opponentShips)
        {
            var hitClusters = new List<List<(int X, int Y)>>();

            var visited = new bool[10, 10];

            foreach (var shot in previousShots
                .Where(s => s.IsHit 
                    && !GridHelper.IsPartOfSunkShip(s.X, s.Y, opponentShips)))
            {
                if (!visited[shot.X, shot.Y])
                {
                    var cluster = new List<(int X, int Y)>();
                    var queue = new Queue<(int X, int Y)>();

                    queue.Enqueue((shot.X, shot.Y));

                    while (queue.Count > 0)
                    {
                        var (cx, cy) = queue.Dequeue();

                        if (visited[cx, cy]) continue;

                        visited[cx, cy] = true;
                        cluster.Add((cx, cy));

                        foreach (var (X, Y) 
                            in GridHelper.GetSideAdjacentCells(cx, cy))
                        {
                            if (GridHelper.IsWithinBounds(X, Y)
                                && !visited[X, Y])
                            {
                                var adjacentShot
                                    = previousShots.FirstOrDefault(
                                        s => s.X == X
                                            && s.Y == Y
                                            && s.IsHit);
                                if (adjacentShot != null
                                    && !GridHelper.IsPartOfSunkShip(
                                        adjacentShot.X,
                                        adjacentShot.Y,
                                        opponentShips))
                                {
                                    queue.Enqueue((X, Y));
                                }
                            }
                        }
                    }

                    if (cluster.Count > 0)
                    {
                        GridHelper.OrderHitCluster(ref cluster);
                        hitClusters.Add(cluster);
                    }
                }
            }

            return hitClusters;
        }

        public static void AdjustProbabilityForSingleHit(
            List<Shot> previousShots,
            int[] probabilityMap,
            (int X, int Y) hit,
            int weight)
        {
            var adjacentCells
                = GridHelper.GetSideAdjacentCells(
                    hit.X,
                    hit.Y);

            foreach (var (X, Y) in adjacentCells)
            {
                IncreaseProbabilityForCell(
                    previousShots, X, Y,
                    probabilityMap, weight);
            }
        }
        public static void IncreaseProbabilityForCell(
            List<Shot> previousShots,
            int x, int y,
            int[] probabilityMap,
            int weight)
        {
            if (GridHelper.IsWithinBounds(x, y)
                && GridHelper.IsCellAvailable(previousShots, x, y))
            {
                probabilityMap[y * 10 + x] += weight;
            }
        }

        public static void AdjustProbabilityForShotAtCells(
            List<Shot> previousShots,
            int[] probabilityMap)
        {
            for (int x = 0; x < 10; x++)
            {
                for (int y = 0; y < 10; y++)
                {
                    if (!GridHelper.IsCellAvailable(previousShots, x, y))
                    {
                        probabilityMap[(y * 10) + x] = 0;
                    }
                }
            }
        }

        public static void AdjustProbabilityForSunkShips(
            List<Ship> opponentShips, int[] probabilityMap)
        {
            foreach (var ship in opponentShips.Where(ship => ship.IsSunk))
            {
                foreach (var coord in ship.Coordinates)
                {
                    var adjacentCells = GridHelper.GetAllAdjacentCells(
                        coord.X, coord.Y);
                    foreach (var (X, Y) in adjacentCells)
                    {
                        if (GridHelper.IsWithinBounds(X, Y))
                        {
                            probabilityMap[(Y * 10) + X] = 0;
                        }
                    }
                }
            }
        }
    }
}

\end{lstlisting}
\end{addmargin}

\newpage
\appendix{Arkusz z wynikami testów oraz wykorzystane zapytania SQL}

Dostępny na płycie dołączonej do pracy.

\newpage
\appendix{Dane zgromadzone w bazie danych oraz skrypty w Pythonie użyte do ich importu}

Dostępny na płycie dołączonej do pracy.

% Używając powyższych spisów jako szablonu,
% możesz tu dodać swój własny wykaz bądź listę,
% np. spis algorytmów.

\end{document} % Dobranoc.
